\chapterbegin

\section[ATS Surveillance Services]{ATS SURVEILLANCE SERVICES}

%References
\note{ADS-contract (ADS-C), at this time used wholly to provide procedural separation, is covered in Chapter 13.}

\subsection[ATS surveillance systems capabilities]{ATS SURVEILLANCE SYSTEMS CAPABILITIES}

\begin{enumnoss}
    \item ATS surveillance systems used in the provision of air traffic services shall have a very high level of reliability, availability and integrity. The possibility of system failures or significant system degradations which may cause complete or partial interruptions of service shall be very remote. Backup facilities shall be provided.
    \note[1]{An ATS surveillance system will normally consist of a number of integrated elements, including sensor(s), data transmission links, data-processing systems and situation displays.}
    \note[2]{Guidance material pertaining to use of radar and to system performance is contained in the \emph{Manual on Testing of Radio Navigation Aids} (Doc 8071), the \emph{Manual on the Secondary Surveillance Radar (SSR) Systems} (Doc 9684) and the \emph{Air Traffic Services Planning Manual} (Doc 9426).}
    \note[3]{Guidance material pertaining to use of ADS-B and MLAT systems and their system performance is contained in Cir 326.}
    \note[4]{Functional and performance requirements pertaining to ATS surveillance systems are contained in \emph{Annex 10} — Aeronautical Telecommunications, \emph{Volume IV} — Surveillance and Collision Avoidance Systems.}
    \item ATS surveillance systems should have the capability to receive, process and display, in an integrated manner, data from all the connected sources.
    \item ATS surveillance systems should be capable of integration with other automated systems used in the provision of ATS, and should provide for an appropriate level of automation with the objectives of improving the accuracy and timeliness of data displayed to the controller and reducing controller workload and the need for verbal coordination between adjacent control positions and ATC units.
    \item ATS surveillance systems should provide for the display of safety-related alerts and warnings, including conflict alert, minimum safe altitude warning, conflict prediction and unintentionally duplicated SSR codes and aircraft identification.
    \item States should, to the extent possible, facilitate the sharing of information derived from ATS surveillance systems in order to extend and improve surveillance coverage in adjacent control areas.
    \item States should, on the basis of regional air navigation agreements, provide for the automated exchange of coordination data relevant to aircraft being provided with ATS surveillance services, and establish automated coordination procedures.
    \item ATS surveillance systems, such as primary surveillance radar (PSR), secondary surveillance radar (SSR), ADS-B and MLAT systems may be used either alone or in combination in the provision of air traffic services, including in the provision of separation between aircraft, provided:

    \begin{enumalph}
        \item reliable coverage exists in the area;
        \item the probability of detection, the accuracy and the integrity of the ATS surveillance system(s) are satisfactory; and
        \item in the case of ADS-B, the availability of data from participating aircraft is adequate.
    \end{enumalph}

    \item PSR systems should be used in circumstances where other ATS surveillance systems alone would not meet the air traffic services requirements.
    \item SSR systems, especially those utilizing monopulse techniques or having Mode S capability or MLAT, may be used alone, including in the provision of separation between aircraft, provided:

    \begin{enumalph}
        \item the carriage of SSR transponders is mandatory within the area; and
        \item identification is established and maintained.
    \end{enumalph}

    \item ADS-B shall only be used for the provision of air traffic control service provided the quality of the information contained in the ADS-B message exceeds the values specified by the appropriate ATS authority.
    \item ADS-B may be used alone, including in the provision of separation between aircraft, provided:

    \begin{enumalph}
        \item identification of ADS-B-equipped aircraft is established and maintained;
        \item the data integrity measure in the ADS-B message is adequate to support the separation minimum;
        \item there is no requirement for detection of aircraft not transmitting ADS-B; and
        \item there is no requirement for determination of aircraft position independent of the position-deter\-mining elements of the aircraft navigation system.
    \end{enumalph}

    \item The provision of ATS surveillance services shall be limited to specified areas of coverage and shall be subject to such other limitations as have been specified by the appropriate ATS authority. Adequate information on the operating methods used shall be published in aeronautical information publications, as well as operating practices and/or equipment limitations having direct effects on the operation of the air traffic services.
    %References
    \note{States will provide information on the area or areas where PSR, SSR, ADS-B and MLAT systems are in use as well as ATS surveillance services and procedures in accordance with Annex 15, 4.1.1 and Appendix 1.}

    \begin{enumnoss}
        \item The provision of ATS surveillance services shall be limited when position data quality degrades below a level specified by the appropriate ATS authority.
    \end{enumnoss}

    \item Where PSR and SSR are required to be used in combination, SSR alone may be used in the event of PSR failure to provide separation between identified transponder-equipped aircraft, provided the accuracy of the SSR position indications has been verified by monitor equipment or other means.
\end{enumnoss}

\subsection[Situation display]{SITUATION DISPLAY}

\begin{enumnoss}
    \item A situation display providing surveillance information to the controller shall, as a minimum, include position indications, map information required to provide ATS surveillance services and, where available, information concerning the identity of the aircraft and the aircraft level.
    \item The ATS surveillance system shall provide for a continuously updated presentation of surveillance information, including position indications.
    \item Position indications may be displayed as:

    \begin{enumalph}
        \item individual position symbols, e.g. PSR, SSR, ADS-B or MLAT symbols, or combined symbols;
        \item PSR blips; and
        \item SSR responses.
    \end{enumalph}

    \item When applicable, distinct symbols should be used for presentation of:

    \begin{enumalph}
        \item unintentionally duplicated SSR codes and/or aircraft identification that are unintentionally duplicated;
        \item predicted positions for a non-updated track; and
        \item plot and track data.
    \end{enumalph}

    \item Where surveillance data quality degrades such that services need to be limited, symbology or other means shall be used to provide the controller with an indication of the condition.
    \item Reserved SSR codes, including 7500, 7600 and 7700, operation of IDENT, ADS-B emergency and/or urgency modes, safety-related alerts and warnings as well as information related to automated coordination shall be presented in a clear and distinct manner, providing for ease of recognition.
    \item Labels associated with displayed targets should be used to provide, in alphanumeric form, information derived from the means of surveillance and, where necessary, the flight data processing system.
    \item Labels shall, as a minimum, include information relating to the identity of the aircraft, e.g. SSR code or aircraft identification and, if available, pressure-altitude-derived level information. This information may be obtained from SSR Mode A, SSR Mode C, SSR Mode S and/or ADS-B.
    \item Labels shall be associated with their position indications in a manner precluding erroneous identification by or confusion on the part of the controller. All label information shall be presented in a clear and concise manner.
\end{enumnoss}

\subsection[Communications]{COMMUNICATIONS}

\begin{enumnoss}
    \item The level of reliability and availability of communications systems shall be such that the possibility of system failures or significant degradations is very remote. Adequate backup facilities shall be provided.
    \note{Guidance material and information pertaining to system reliability and availability are contained in Annex 10, Volume I, and the \emph{Air Traffic Services Planning Manual} (Doc 9426).}
    \item Direct pilot-controller communications shall be established prior to the provision of ATS surveillance services, unless special circumstances, such as emergencies, dictate otherwise.
\end{enumnoss}

\subsection[Provision of ATS surveillance services]{PROVISION OF ATS SURVEILLANCE SERVICES}

\begin{enumnoss}
    \item Information derived from ATS surveillance systems, including safety-related alerts and warnings such as conflict alert and minimum safe altitude warning, should be used to the extent possible in the provision of air traffic control service in order to improve capacity and efficiency as well as to enhance safety.
    \item The number of aircraft simultaneously provided with ATS surveillance services shall not exceed that which can safely be handled under the prevailing circumstances, taking into account:

    \begin{enumalph}
        \item the structural complexity of the control area or sector concerned;
        \item the functions to be performed within the control area or sector concerned;
        \item assessments of controller workloads, taking into account different aircraft capabilities, and sector capacity; and
        \item the degree of technical reliability and availability of the primary and backup communications, navigation and surveillance systems, both in the aircraft and on the ground.
    \end{enumalph}
\end{enumnoss}

\subsection[Use of SSR transponders and ADS-B transmitters]{USE OF SSR TRANSPONDERS AND ADS-B TRANSMITTERS}

\subsubsection{General}

To ensure the safe and efficient use of ATS surveillance services, pilots and controllers shall strictly adhere to published operating procedures and standard radiotelephony phraseology shall be used. The correct setting of transponder codes and/or aircraft identification shall be ensured at all times.

\subsubsection{SSR code management}

\begin{enumerate}
    \item Codes 7700, 7600 and 7500 shall be reserved internationally for use by pilots encountering a state of emergency, radiocommunication failure or unlawful interference, respectively.

    % 8.5.2.2
    % 8.5.2.3
\end{enumerate}

\subsubsection{Operation of SSR transponders}

\note{SSR transponder operating procedures are contained in \emph{Procedures for Air Navigation Services — Aircraft Operations} (PANS-OPS, Doc 8168), Volume I, Part III, Section 3.}

\begin{enumerate}
    \item When it is observed that the Mode A code shown on the situation display is different to what has been assigned to the aircraft, the pilot shall be requested to confirm the code selected and, if the situation warrants (e.g. not being a case of unlawful interference), to reselect the correct code.
    \item If the discrepancy between assigned and displayed Mode A codes still persists, the pilot may be requested to stop the operation of the aircraft’s transponder. The next control position and any other affected unit using SSR and/or MLAT in the provision of ATS shall be informed accordingly.
    %Review
    \item Aircraft equipped with Mode S having an aircraft identification feature shall transmit the aircraft identification as specified in Item 7 of the ICAO flight plan or, when no flight plan has been filed, the aircraft registration.
    \note{All Mode S-equipped aircraft engaged in international civil aviation are required to have an aircraft identification feature (Annex 10, Volume IV, Chapter 2, 2.1.5.2, refers).}
    \item Whenever it is observed on the situation display that the aircraft identification transmitted by a Mode S-equipped aircraft is different from that expected from the aircraft, the pilot shall be requested to confirm and, if necessary, re-enter the correct aircraft identification.
    \item If, following confirmation by the pilot that the correct aircraft identification has been set on the Mode S identification feature, the discrepancy continues to exist, the following actions shall be taken by the controller:

    \begin{enumalph}
        \item inform the pilot of the persistent discrepancy;
        \item where possible, correct the label showing the aircraft identification on the situation display; and
        \item notify the erroneous aircraft identification transmitted by the aircraft to the next control position and any other interested unit using Mode S for identification purposes.
    \end{enumalph}
\end{enumerate}

\subsubsection{Operation of ADS-B transmitters}

\begin{noteev}
    \note[1]{To indicate that it is in a state of emergency or to transmit other urgent information, an aircraft equipped with ADS-B might operate the emergency and/or urgency mode as follows:}
    \begin{enumalph}
        \item emergency;
        \item communication failure;
        \item unlawful interference;
        \item minimum fuel; and/or
        \item medical.
    \end{enumalph}
    %References
    \note[2]{Some aircraft equipped with first generation ADS-B avionics do not have the capability described in Note 1 above and only have the capability to transmit a general emergency alert regardless of the code selected by the pilot.}
\end{noteev}

\begin{enumerate}
    %Review
    \item Aircraft equipped with ADS-B having an aircraft identification feature shall transmit the aircraft identification as specified in Item 7 of the ICAO flight plan or, when no flight plan has been filed, the aircraft registration.
    \item Whenever it is observed on the situation display that the aircraft identification transmitted by an ADS-B-equipped aircraft is different from that expected from the aircraft, the pilot shall be requested to confirm and, if necessary, re-enter the correct aircraft identification.
    \item If, following confirmation by the pilot that the correct aircraft identification has been set on the ADS-B identification feature, the discrepancy continues to exist, the following actions shall be taken by the controller:

    \begin{enumalph}
        \item inform the pilot of the persistent discrepancy;
        \item where possible, correct the label showing the aircraft identification on the situation display; and
        \item notify the next control position and any other unit concerned of the erroneous aircraft identification transmitted by the aircraft.
    \end{enumalph}
\end{enumerate}

\subsubsection[Level information based on the use of pressure-altitude information]{Level information based on the \\ use of pressure-altitude information}

\begin{enumeratesc}
    \item \textsc{Verification of level information}
    \begin{enumerate}
        \item The tolerance value used to determine that pressure-altitude-derived level information displayed to the controller is accurate shall be $\pm$200 ft in RVSM airspace. In other airspace, it shall be $\pm$300 ft, except that the appropriate ATS authority may specify a smaller criterion, but not less than $\pm$200 ft, if this is found to be more practical. Geometric height information shall not be used for separation.
        \item Verification of pressure-altitude-derived level information displayed to the controller shall be effected at least once by each suitably equipped ATC unit on initial contact with the aircraft concerned or, if this is not feasible, as soon as possible thereafter. The verification shall be effected by simultaneous comparison with altimeter-derived level information received from the same aircraft by radiotelephony. The pilot of the aircraft whose pressure-altitude-derived level information is within the approved tolerance value need not be advised of such verification. Geometric height information shall not be used to determine if altitude differences exist.
        \item If the displayed level information is not within the approved tolerance value or when a discrepancy in excess of the approved tolerance value is detected subsequent to verification, the pilot shall be advised accordingly and requested to check the pressure setting and confirm the aircraft’s level.
        \item If, following confirmation of the correct pressure setting the discrepancy continues to exist, the following action should be taken according to circumstances:

        \begin{enumalph}
            \item request the pilot to stop Mode C or ADS-B altitude data transmission, provided this does not cause the loss of position and identity information, and notify the next control positions or ATC unit concerned with the aircraft of the action taken; or
            \item inform the pilot of the discrepancy and request that the relevant operation continue in order to prevent loss of position and identity information of the aircraft and, when authorized by the appropriate ATS authority, override the label-displayed level information with the reported level. Notify the next control position or ATC unit concerned with the aircraft of the action taken.
        \end{enumalph}
    \end{enumerate}

    \item \textsc{Determination of level occupancy}
    \begin{enumerate}
        \item The criterion which shall be used to determine that a specific level is occupied by an aircraft shall be $\pm$200 ft in RVSM airspace. In other airspace, it shall be $\pm$300 ft, except that the appropriate ATS authority may specify a smaller criterion, but not less than $\pm$200 ft, if this is found to be more practical.
        \note{For a brief explanation of the considerations underlying this value, see the \emph{Air Traffic Services Planning Manual} (Doc 9426).}
        %References
        \item \textit{Aircraft maintaining a level.} An aircraft is considered to be maintaining its assigned level as long as the pressure-altitude-derived level information indicates that it is within the appropriate tolerances of the assigned level, as specified in 8.5.5.2.1.
        \item \textit{Aircraft vacating a level.} An aircraft cleared to leave a level is considered to have commenced its manoeuvre and vacated the previously occupied level when the pressure-altitude-derived level information indicates a change of more than 300 ft in the anticipated direction from its previously assigned level.
        \item \textit{Aircraft passing a level in climb or descent.} An aircraft in climb or descent is considered to have crossed a level when the pressure-altitude-derived level information indicates that it has passed this level in the required direction by more than 300 ft.
        %References
        \item \textit{Aircraft reaching a level.} An aircraft is considered to have reached the level to which it has been cleared when the elapsed time of three display updates, three sensor updates or 15 seconds, whichever is the greater, has passed since the pressure-altitude-derived level information has indicated that it is within the appropriate tolerances of the assigned level, as specified in 8.5.5.2.1.
        \item Intervention by a controller shall only be required if differences in level information between that displayed to the controller and that used for control purposes are in excess of the values stated above.
    \end{enumerate}
\end{enumeratesc}

\subsection[General procedures]{GENERAL PROCEDURES}

%Numbering
% 8.6.1

\subsubsection{Identification of aircraft}

\begin{enumeratesc}
    \item \textsc{Establishment of identification}
    \begin{enumerate}
        \item Before providing an ATS surveillance service to an aircraft, identification shall be established and the pilot informed. Thereafter, identification shall be maintained until termination of the ATS surveillance service.
        \item If identification is subsequently lost, the pilot shall be informed accordingly and, when applicable, appropriate instructions issued.
        %References
        \item Identification shall be established by at least one of the methods specified in 8.6.2.2, 8.6.2.3, 8.6.2.4 and 8.6.2.5.
    \end{enumerate}

    \item \textsc{ADS-B identification procedures}
    \begin{enumempty}
        \item Where ADS-B is used for identification, aircraft may be identified by one or more of the following procedures:
    \end{enumempty}
    \begin{enumalph}
        \item direct recognition of the aircraft identification in an ADS-B label;
        %References
        \item transfer of ADS-B identification (see 8.6.3); and
        \item observation of compliance with an instruction to TRANSMIT ADS-B IDENT.
        \note[1]{Some aircraft equipped with first generation ADS-B avionics do not have the capability of squawking IDENT while the emergency and/or urgency mode is selected.}
        \note[2]{In automated systems, the “IDENT” feature may be presented in different ways, e.g. as a flashing of all or part of the position indication and associated label.}
    \end{enumalph}
    
    \item \textsc{SSR and/or MLAT identification procedures}
    \begin{enumerate}
        \item Where SSR and/or MLAT is used for identification, aircraft may be identified by one or more of the following procedures:
        
        \begin{enumalph}
            \item recognition of the aircraft identification in an SSR and/or MLAT label;
            %References
            \note{The use of this procedure requires that the code/call sign correlation is achieved successfully, taking into account the Note following b) below.}
            \item recognition of an assigned discrete code, the setting of which has been verified, in an SSR and/or MLAT label;
            %References
            \note{The use of this procedure requires a system of code assignment which ensures that each aircraft in a given portion of airspace is assigned a discrete code (see 8.5.2.2.7).}
            \item direct recognition of the aircraft identification of a Mode S-equipped aircraft in an SSR and/or MLAT label;
            \note{The aircraft identification feature available in Mode S transponders provides the means to identify directly individual aircraft on situation displays and thus offers the potential to eliminate ultimately the recourse to Mode A discrete codes for individual identification. This elimination will only be achieved in a progressive manner depending on the state of deployment of suitable ground and airborne installations.}
            %References
            \item by transfer of identification (see 8.6.3);
            \item observation of compliance with an instruction to set a specific code;
            \item observation of compliance with an instruction to squawk IDENT.
            \note[1]{In automated radar systems, the “IDENT” feature may be presented in different ways, e.g. as a flashing of all or part of the position indication and associated label.}
            \note[2]{Garbling of transponder replies may produce “IDENT”-type of indications. Nearly simultaneous “IDENT” transmissions within the same area may give rise to errors in identification.}
        \end{enumalph}
        
        \item When a discrete code has been assigned to an aircraft, a check shall be made at the earliest opportunity to ensure that the code set by the pilot is identical to that assigned for the flight. Only after this check has been made shall the discrete code be used as a basis for identification.
    \end{enumerate}

    \item \textsc{PSR identification procedures}
    \begin{enumerate}
        \item Where PSR is used for identification, aircraft may be identified by one or more of the following procedures:
        
        \begin{enumalph}
            \item by correlating a particular radar position indication with an aircraft reporting its position over, or as bearing and distance from, a point shown on the situation display, and by ascertaining that the track of the particular radar position is consistent with the aircraft path or reported heading;
            
            \begin{noteev}
                \note[1]{Caution must be exercised when employing this method since a position reported in relation to a point may not coincide precisely with the radar position indication of the aircraft on the situation display. The appropriate ATS authority may, therefore, prescribe additional conditions for the application of this method, e.g.:}
                \begin{enumroman}
                    \item a level or levels above which this method may not be applied in respect of specified navigation aids; or
                    \item a distance from the radar site beyond which this method may not be applied.
                \end{enumroman}
                \note[2]{The term “a point” refers to a geographical point suitable for the purposes of identification. It is normally a reporting point defined by reference to a radio navigation aid or aids.}
            \end{noteev}
            
            \item by correlating an observed radar position indication with an aircraft which is known to have just departed, provided that the identification is established within 1 NM from the end of the runway used. Particular care should be taken to avoid confusion with aircraft holding over or overflying the aerodrome, or with aircraft departing from or making a missed approach over adjacent runways;
            %References
            \item by transfer of identification (see 8.6.3);
            \item by ascertaining the aircraft heading, if circumstances require, and following a period of track observation:
            
            \begin{enumerate}[label=---,labelsep=0.3cm,leftmargin=*,labelindent=0pt]
                \item instructing the pilot to execute one or more changes of heading of 30 degrees or more and correlating the movements of one particular radar position indication with the aircraft’s acknowledged execution of the instructions given; or
                \item correlating the movements of a particular radar position indication with manoeuvres currently executed by an aircraft having so reported.
            \end{enumerate}
            
            \noindent When using these methods, the controller shall:
            
            \begin{enumroman}
                \item verify that the movements of not more than one radar position indication correspond with those of the aircraft; and
                \item ensure that the manoeuvre(s) will not carry the aircraft outside the coverage of the radar or the situation display.
            \end{enumroman}
        
            \note[1]{Caution must be exercised when employing these methods in areas where route changes normally take place.}
            %References
            \note[2]{With reference to ii) above, see also 8.6.5.1 regarding vectoring of controlled aircraft.}
        \end{enumalph}
    
        \item Use may be made of direction-finding bearings to assist in identification of an aircraft. This method, however, shall not be used as the sole means of establishing identification, unless so prescribed by the appropriate ATS authority for particular cases under specified conditions.
    \end{enumerate}

    \item \textsc{Additional identification method}
    \begin{enumempty}
        \item When two or more position indications are observed in close proximity, or are observed to be making similar movements at the same time, or when doubt exists as to the identity of a position indication for any other reason, changes of heading should be prescribed or repeated as many times as necessary, or additional methods of identification should be employed, until all risk of error in identification is eliminated.
    \end{enumempty}    
\end{enumeratesc}

\subsubsection{Transfer of identification}

\begin{enumerate}
    \item Transfer of identification from one controller to another should only be attempted when it is considered that the aircraft is within the accepting controller’s surveillance coverage.
    \item Transfer of identification shall be effected by one of the following methods:
    
    \begin{enumalph}
        \item designation of the position indication by automated means, provided that only one position indication is thereby indicated and there is no possible doubt of correct identification;
        \item notification of the aircraft’s discrete SSR code or aircraft address;
        %References
        \note[1]{The use of a discrete SSR code requires a system of code assignment which ensures that each aircraft in a given portion of airspace is assigned a discrete code (see 8.5.2.2.7).}
        \note[2]{Aircraft address would be expressed in the form of the alphanumerical code of six hexadecimal characters.}
        \item notification that the aircraft is SSR Mode S-equipped with an aircraft identification feature when SSR Mode S coverage is available;
        \item notification that the aircraft is ADS-B-equipped with an aircraft identification feature when compatible ADS-B coverage is available;
        \item direct designation (pointing with the finger) of the position indication, if the two situation displays are adjacent, or if a common “conference” type of situation display is used;
        \note{Attention must be given to any errors which might occur due to parallax effects.}
        \item designation of the position indication by reference to, or in terms of bearing and distance from, a geographical position or navigational facility accurately indicated on both situation displays, together with the track of the observed position indication if the route of the aircraft is not known to both controllers;
        
        \begin{noteev}
            \note{Caution must be exercised before transferring identification using this method, particularly if other position indications are observed on similar headings and in close proximity to the aircraft under control. Inherent radar deficiencies, such as inaccuracies in bearing and distance of the radar position indications displayed on individual situation displays and parallax errors, may cause the indicated position of an aircraft in relation to the known point to differ between the two situation displays. The appropriate ATS authority may, therefore, prescribe additional conditions for the application of this method, e.g.:}
            \begin{enumroman}
                \item a maximum distance from the common reference point used by the two controllers; and
                \item a maximum distance between the position indication as observed by the accepting controller and the one stated by the transferring controller.
            \end{enumroman}
        \end{noteev}

        \item where applicable, issuance of an instruction to the aircraft by the transferring controller to change SSR code and the observation of the change by the accepting controller; or
        \item issuance of an instruction to the aircraft by the transferring controller to squawk/transmit IDENT and observation of this response by the accepting controller.
        %References
        \note{Use of procedures g) and h) requires prior coordination between the controllers, since the indications to be observed by the accepting controller are of short duration.}
    \end{enumalph}
\end{enumerate}

\subsubsection{Position information}

\begin{enumerate}
    \item An aircraft provided with ATS surveillance service should be informed of its position in the following circumstances:
    
    \begin{enumalph}
        \item upon identification, except when the identification is established:
        
        \begin{enumroman}
            \item based on the pilot’s report of the aircraft position or within one nautical mile of the runway upon departure and the observed position on the situation display is consistent with the aircraft’s time of departure; or
            \item by use of ADS-B aircraft identification, Mode S aircraft identification or assigned discrete SSR codes and the location of the observed position indication is consistent with the current flight plan of the aircraft; or
            \item by transfer of identification;
        \end{enumroman}

        \item when the pilot requests this information;
        \item when a pilot’s estimate differs significantly from the controller’s estimate based on the observed position;
        %References
        \item when the pilot is instructed to resume own navigation after vectoring if the current instructions had diverted the aircraft from a previously assigned route (see 8.6.5.5);
        \item immediately before termination of ATS surveillance service, if the aircraft is observed to deviate from its intended route.
    \end{enumalph}

    \item Position information shall be passed to aircraft in one of the following forms:
    
    \begin{enumalph}
        \item as a well-known geographical position;
        \item magnetic track and distance to a significant point, an en-route navigation aid, or an approach aid;
        \item direction (using points of the compass) and distance from a known position;
        \item distance to touchdown, if the aircraft is on final approach; or
        \item distance and direction from the centre line of an ATS route.
    \end{enumalph}

    \item Whenever practicable, position information shall relate to positions or routes pertinent to the navigation of the aircraft concerned and shown on the situation display map.
    \item When so informed, the pilot may omit position reports at compulsory reporting points or report only over those reporting points specified by the air traffic services unit concerned. Unless automated position reporting is in effect (e.g. ADS-C), pilots shall resume voice or CPDLC position reporting:
    
    \begin{enumalph}
        \item when so instructed;
        \item when advised that the ATS surveillance service has been terminated; or
        \item when advised that identification is lost.
    \end{enumalph}
\end{enumerate}

\subsubsection{Vectoring}

\begin{enumerate}
    \item Vectoring shall be achieved by issuing to the pilot specific headings which will enable the aircraft to maintain the desired track. When vectoring an aircraft, a controller shall comply with the following:
    
    \begin{enumalph}
        \item whenever practicable, the aircraft shall be vectored along tracks on which the pilot can monitor the aircraft position with reference to pilot-interpreted navigation aids (this will minimize the amount of navigational assistance required and alleviate the consequences resulting from an ATS surveillance system failure);
        \item when an aircraft is given its initial vector diverting it from a previously assigned route, the pilot shall be informed what the vector is to accomplish, and the limit of the vector shall be specified (e.g. to ... position, for ... approach);
        \item except when transfer of control is to be effected, aircraft shall not be vectored closer than 2.5 NM or, where the minimum permissible separation is greater than 5 NM, a distance equivalent to one-half of the prescribed separation minimum, from the limit of the airspace for which the controller is responsible, unless local arrangements have been made to ensure that separation will exist with aircraft operating in adjoining areas;
        \item controlled flights shall not be vectored into uncontrolled airspace except in the case of emergency or in order to circumnavigate adverse meteorological conditions (in which case the pilot should be so informed), or at the specific request of the pilot; and
        \item when an aircraft has reported unreliable directional instruments, the pilot shall be requested, prior to the issuance of manoeuvring instructions, to make all turns at an agreed rate and to carry out the instructions immediately upon receipt.
    \end{enumalph}

    \item When vectoring an IFR flight and when giving an IFR flight a direct routing which takes the aircraft off an ATS route, the controller shall issue clearances such that the prescribed obstacle clearance will exist at all times until the aircraft reaches the point where the pilot will resume own navigation. When necessary, the relevant minimum vectoring altitude shall include a correction for low temperature effect.
    %References
    \note[1]{When an IFR flight is being vectored, the pilot may be unable to determine the aircraft’s exact position in respect to obstacles in this area and consequently the altitude which provides the required obstacle clearance. Detailed obstacle clearance criteria are contained in PANS-OPS (Doc 8168), Volumes I and II. See also 8.6.8.2.}
    \note[2]{It is the responsibility of the ATS authority to provide the controller with minimum altitudes corrected for temperature effect.}
    \item Whenever possible, minimum vectoring altitudes should be sufficiently high to minimize activation of aircraft ground proximity warning systems.
    \note{Activation of such systems will induce aircraft to pull up immediately and climb steeply to avoid hazardous terrain, possibly compromising separation between aircraft.}
    %Numbering
    % 8.6.5.4
    %References
    \item In terminating vectoring of an aircraft, the controller shall instruct the pilot to resume own navigation, giving the pilot the aircraft’s position and appropriate instructions, as necessary, in the form prescribed in 8.6.4.2 b), if the current instructions had diverted the aircraft from a previously assigned route.
\end{enumerate}

\subsubsection{Navigation assistance}

\begin{enumerate}
    \item An identified aircraft observed to deviate significantly from its intended route or designated holding pattern shall be advised accordingly. Appropriate action shall also be taken if, in the opinion of the controller, such deviation is likely to affect the service being provided.
    \item The pilot of an aircraft requesting navigation assistance from an air traffic control unit providing ATS surveillance services shall state the reason (e.g. to avoid areas of adverse weather or unreliable navigational instruments) and shall give as much information as possible in the circumstances.
\end{enumerate}

\subsubsection{Interruption or termination of ATS surveillance service}

\begin{enumerate}
    \item An aircraft which has been informed that it is provided with ATS surveillance service should be informed immediately when, for any reason, the service is interrupted or terminated.
    \note{The transition of an aircraft across adjoining areas of radar and/or ADS-B and/or MLAT systems coverage will not normally constitute an interruption or termination of the ATS surveillance service.}
    \item When the control of an identified aircraft is to be transferred to a control sector that will provide the aircraft with procedural separation, the transferring controller shall ensure that appropriate procedural separation is established between that aircraft and any other controlled aircraft before the transfer is effected.
\end{enumerate}

\subsubsection{Minimum levels}

\begin{enumerate}
    \item The controller shall at all times be in possession of full and up-to-date information regarding:
    
    \begin{enumalph}
        \item established minimum flight altitudes within the area of responsibility;
        %References
        \item the lowest usable flight level or levels determined in accordance with Chapters 4 and 5; and
        \item established minimum altitudes applicable to procedures based on tactical vectoring.
    \end{enumalph}

    \item Unless otherwise specified by the appropriate ATS authority, minimum altitudes for procedures based on tactical vectoring with any ATS surveillance system shall be determined using the criteria applicable to tactical radar vectoring.
    \note{Criteria for the determination of minimum altitudes applicable to procedures based on tactical radar
    vectoring are contained in Procedures for \emph{Air Navigation Services — Aircraft Operations} (PANS-OPS, Doc 8168), Volume II.}
\end{enumerate}

\subsubsection{Information regarding adverse weather}

\begin{enumerate}
    \item Information that an aircraft appears likely to penetrate an area of adverse weather should be issued in sufficient time to permit the pilot to decide on an appropriate course of action, including that of requesting advice on how best to circumnavigate the adverse weather area, if so desired.
    \note{Depending on the capabilities of the ATS surveillance system, areas of adverse weather may not be presented on the situation display. An aircraft’s weather radar will normally provide better detection and definition of adverse weather than radar sensors in use by ATS.}
    \item In vectoring an aircraft for circumnavigating any area of adverse weather, the controller should ascertain that the aircraft can be returned to its intended or assigned flight path within the coverage of the ATS surveillance system and, if this does not appear possible, inform the pilot of the circumstances.
    \note{Attention must be given to the fact that under certain circumstances the most active area of adverse weather may not be displayed.}
\end{enumerate}

\subsection[Use of ATS surveillance systems in the air traffic control service]{USE OF ATS SURVEILLANCE SYSTEMS IN THE \\ AIR TRAFFIC CONTROL SERVICE}

%References
\note{The procedures in this Section are general procedures applicable when an ATS surveillance system is used in the provision of area control service or approach control service. Additional procedures applicable in the provision of approach control service are detailed in Section 8.9.}

\subsubsection{Functions}

The information provided by ATS surveillance systems and presented on a situation display may be used to perform the following functions in the provision of air traffic control service:

\begin{enumalph}
    \item provide ATS surveillance services as necessary in order to improve airspace utilization, reduce delays, provide for direct routings and more optimum flight profiles, as well as to enhance safety;
    \item provide vectoring to departing aircraft for the purpose of facilitating an expeditious and efficient departure flow and expediting climb to cruising level;
    \item provide vectoring to aircraft for the purpose of resolving potential conflicts;
    \item provide vectoring to arriving aircraft for the purpose of establishing an expeditious and efficient approach sequence;
    \item provide vectoring to assist pilots in their navigation, e.g. to or from a radio navigation aid, away from or around areas of adverse weather;
    \item provide separation and maintain normal traffic flow when an aircraft experiences communication failure within the area of coverage;
    \item maintain flight path monitoring of air traffic;
    \note{Where tolerances regarding such matters as adherence to track, speed or time have been prescribed by the appropriate ATS authority, deviations are not considered significant until such tolerances are exceeded.}
    \item when applicable, maintain a watch on the progress of air traffic, in order to provide a procedural controller with:
    
    \begin{enumroman}
        \item improved position information regarding aircraft under control;
        \item supplementary information regarding other traffic; and
        \item information regarding any significant deviations by aircraft from the terms of their respective air traffic control clearances, including their cleared routes as well as levels, when appropriate.
    \end{enumroman}
\end{enumalph}

\subsubsection{Separation application}

\note{Factors which the controller using an ATS surveillance system must take into account in determining the spacing to be applied in particular circumstances in order to ensure that the separation minimum is not infringed include aircraft relative headings and speeds, ATS surveillance system technical limitations, controller workload and any difficulties caused by communication congestion. Guidance material on this subject is contained in the \emph{Air Traffic Services Planning Manual} (Doc 9426).}

\begin{enumerate}
    %References
    \item Except as provided for in 8.7.2.8, 8.7.2.9 and 8.8.2.2, the separation minima specified in 8.7.3 shall only be applied between identified aircraft when there is reasonable assurance that identification will be maintained.
    \item When control of an identified aircraft is to be transferred to a control sector that will provide the aircraft with procedural separation, such separation shall be established by the transferring controller before the aircraft reaches the limits of the transferring controller’s area of responsibility, or before the aircraft leaves the relevant area of surveillance coverage.
    \item When authorized by the appropriate ATS authority, separation based on the use of ADS-B, SSR and/or MLAT, and/or PSR position symbols and/or PSR blips shall be applied so that the distance between the centres of the position symbols and/or PSR blips, representing the positions of the aircraft concerned, is never less than a prescribed minimum.
    \item Separation based on the use of PSR blips and SSR responses shall be applied so that the distance between the centre of the PSR blip and the nearest edge of the SSR response (or centre, when authorized by the appropriate ATS authority) is never less than a prescribed minimum.
    \item Separation based on the use of ADS-B position symbols and SSR responses shall be applied so that the distance between the centre of the ADS-B position symbol and the nearest edge of the SSR response (or the centre, when authorized by the appropriate ATS authority) is never less than a prescribed minimum.
    \item Separation based on the use of SSR responses shall be applied so that the distance between the closest edges of the SSR responses (of the centres, when authorized by the appropriate ATS authority) is never less than a prescribed minimum.
    \item In no circumstances shall the edges of the position indications touch or overlap unless vertical separation is applied between the aircraft concerned, irrespective of the type of position indication displayed and separation minimum applied.
    %References
    \item In the event that the controller has been notified of a controlled flight entering or about to enter the airspace within which the separation minima specified in 8.7.3 is applied, but has not identified the aircraft, the controller may, if so prescribed by the appropriate ATS authority, continue to provide an ATS surveillance service to identified aircraft provided that:
    
    \begin{enumalph}
        \item reasonable assurance exists that the unidentified controlled flight will be identified using SSR and/or ADS-B and/or MLAT or the flight is being operated by an aircraft of a type which may be expected to give an adequate return on primary radar in the airspace within which the separation is applied; and
        \item the separation is maintained between identified flights and any other observed ATS surveillance system position indications until either the unidentified controlled flight has been identified or procedural separation has been established.
    \end{enumalph}

    %References
    \item The separation minima specified in 8.7.3 may be applied between an aircraft taking off and a preceding departing aircraft or other identified traffic provided there is reasonable assurance that the departing aircraft will be identified within 1 NM from the end of the runway, and that, at the time, the required separation will exist.
    \item The separation minima specified in 8.7.3 shall not be applied between aircraft holding over the same holding fix. Application of ATS surveillance system separation minima based on radar and/or ADS-B and/or MLAT systems between holding aircraft and other flights shall be subject to requirements and procedures prescribed by the appropriate ATS authority.
\end{enumerate}

\subsubsection{Separation minima based on ATS surveillance systems}

\begin{enumerate}
    %References
    \item Unless otherwise prescribed in accordance with 8.7.3.2, 8.7.3.3 or 8.7.3.4, or Chapter 6 (with respect to independent and dependent parallel approaches), the horizontal separation minimum based on radar and/or ADS-B and/or MLAT systems shall be 5.0 NM.
    \item The separation minimum in 8.7.3.1 may, if so prescribed by the appropriate ATS authority, be reduced, but not below:
    
    \begin{enumalph}
        \item 3.0 NM when radar and/or ADS-B and/or MLAT systems’ capabilities at a given location so permit; and
        \item 2.5 NM between succeeding aircraft which are established on the same final approach track within 10 NM of the runway threshold. A reduced separation minimum of 2.5 NM may be applied, provided:
        
        \begin{enumroman}[labelsep=0.1cm]
            \item the average runway occupancy time of landing aircraft is proven, by means such as data collection and statistical analysis and methods based on a theoretical model, not to exceed 50 seconds;
            \item braking action is reported as good and runway occupancy times are not adversely affected by runway contaminants such as slush, snow or ice;
            \item an ATS surveillance system with appropriate azimuth and range resolution and an update rate of 5 seconds or less is used in combination with suitable displays;
            \item the aerodrome controller is able to observe, visually or by means of surface movement radar (SMR), MLAT system or a surface movement guidance and control system (SMGCS), the runway-in-use and associated exit and entry taxiways;
            %References
            \item distance-based wake turbulence separation minima in 8.7.3.4, or as may be prescribed by the appropriate ATS authority (e.g. for specific aircraft types), do not apply;
            \item aircraft approach speeds are closely monitored by the controller and when necessary adjusted so as to ensure that separation is not reduced below the minimum;
            \item aircraft operators and pilots have been made fully aware of the need to exit the runway in an expeditious manner whenever the reduced separation minimum on final approach is applied; and
            \item procedures concerning the application of the reduced minimum are published in AIPs.
        \end{enumroman}
    \end{enumalph}

    \item The separation minimum or minima based on radar and/or ADS-B and/or MLAT systems to be applied shall be prescribed by the appropriate ATS authority according to the capability of the particular ATS surveillance system or sensor to accurately identify the aircraft position in relation to the centre of a position symbol, PSR blip, SSR response and taking into account factors which may affect the accuracy of the ATS surveillance system-derived information, such as aircraft range from the radar site and the range scale of the situation display in use.
    %References
    \item The following distance-based wake turbulence separation minima shall be applied to aircraft being provided with an ATS surveillance service in the approach and departure phases of flight in the circumstances given in 8.7.3.4.1.

    \begin{tablecenter}{ccc}
        \toprule
        \addlinespace[2mm]
        \multicolumn{2}{c}{\itshape Aircraft category} & \\
        \itshape \makecell{Preceding \\ aircraft} & \itshape \makecell{Succeeding \\ aircraft} & \itshape \makecell{Distance-based \\ wake turbulence \\ separation minima} \\
        \addlinespace[1mm]
        \midrule
        \addlinespace[2mm]
        \multirow{3}{*}{HEAVY} & HEAVY & 4.0 NM \\
        & MEDIUM & 5.0 NM \\
        & LIGHT & 6.0 NM \\
        \addlinespace[3mm]
        MEDIUM & LIGHT & 5.0 NM \\
        \addlinespace[1mm]
        \bottomrule
    \end{tablecenter}      

    %References
    \note{The provisions governing wake turbulence aircraft categorization are set forth in Chapter 4, Section 4.9.}

    \begin{enumerate}
        %References
        \item The minima set out in 8.7.3.4 shall be applied when:
        
        \begin{enumalph}
            \item an aircraft is operating directly behind another aircraft at the same altitude or less than 1 000 ft below; or
            \item both aircraft are using the same runway, or parallel runways separated by less than 2 500 ft; or
            \item an aircraft is crossing behind another aircraft, at the same altitude or less than 1 000 ft below.
        \end{enumalph}

        %References
        \note{See Figures 8-1A and 8-1B.}

        %Figure
    \end{enumerate}
\end{enumerate}

\subsubsection{Transfer of control}

\begin{enumerate}
    \item Where an ATS surveillance service is being provided, transfer of control should be effected, whenever practicable, so as to enable the uninterrupted provision of the ATS surveillance service.
    \item Where SSR and/or ADS-B and/or MLAT is used and the display of position indications with associated labels is provided for, transfer of control of aircraft between adjacent control positions or between adjacent ATC units may be effected without prior coordination, provided that:
    
    \begin{enumalph}
        \item updated flight plan information on the aircraft about to be transferred, including the discrete assigned SSR code or, with respect to Mode S and ADS-B, the aircraft identification, is provided to the accepting controller prior to transfer;
        \item the ATS surveillance system coverage provided to the accepting controller is such that the aircraft concerned is presented on the situation display before the transfer is effected and is identified on, but preferably before, receipt of the initial call;
        \item when the controllers are not physically adjacent, two-way direct speech facilities, which permit communications to be established instantaneously, are available between them at all times;
        \note{“Instantaneous” refers to communications which effectively provide for immediate access between controllers.}
        \item the transfer point or points and all other conditions of application, such as direction of flight, specified levels, transfer of communication points, and especially an agreed minimum separation between aircraft, including that applicable to succeeding aircraft on the same route, about to be transferred as observed on the situation display, have been made the subject of specific instructions (for intra-unit transfer) or of a specific letter of agreement between two adjacent ATC units;
        \item the instructions or letter of agreement specify explicitly that the application of this type of transfer of control may be terminated at any time by the accepting controller, normally with an agreed advance notice;
        \item the accepting controller is informed of any level, speed or vectoring instructions given to the aircraft prior to its transfer and which modify its anticipated flight progress at the point of transfer.
    \end{enumalph}

    %References
    \item The minimum agreed separation between aircraft about to be transferred (8.7.4.2 d) refers) and the advance notice (8.7.4.2 e) refers) shall be determined taking into account all relevant technical, operational and other circumstances. If circumstances arise in which these agreed conditions can no longer be satisfied, controllers shall revert to the procedure in 8.7.4.4 until the situation is resolved.
    \item Where primary radar is being used, and where another type of ATS surveillance system is employed but the
    provisions of 8.7.4.2 are not applied, the transfer of control of aircraft between adjacent control positions or between two adjacent ATS units may be effected, provided that:

    \begin{enumalph}
        \item identification has been transferred to or has been established directly by the accepting controller;
        \item when the controllers are not physically adjacent, two-way direct-speech facilities between them are at all times available which permit communications to be established instantaneously;
        \item separation from other controlled flights conforms to the minima authorized for use during transfer of control between the sectors or units concerned;
        \item the accepting controller is informed of any level, speed or vectoring instructions applicable to the aircraft at the point of transfer;
        \item radiocommunication with the aircraft is retained by the transferring controller until the accepting controller has agreed to assume responsibility for providing the ATS surveillance service to the aircraft. Thereafter, the aircraft should be instructed to change over to the appropriate channel and from that point is the responsibility of the accepting controller.
    \end{enumalph}
\end{enumerate}

\subsubsection{Speed control}

Subject to conditions specified by the appropriate ATS authority, including consideration of aircraft performance limitations, a controller may, in order to facilitate sequencing or to reduce the need for vectoring, request aircraft to adjust their speed in a specified manner.
%References
\note{Procedures for speed control instructions are contained in Chapter 4, Section 4.6.}

\subsection[Emergencies, hazards and equipment failures]{EMERGENCIES, HAZARDS AND EQUIPMENT FAILURES}

%References
\note{See also Chapter 15.}

\subsubsection{Emergencies}

\begin{enumerate}
    \item In the event of an aircraft in, or appearing to be in, any form of emergency, every assistance shall be provided by the controller, and the procedures prescribed herein may be varied according to the situation.
    \item The progress of an aircraft in emergency shall be monitored and (whenever possible) plotted on the situation display until the aircraft passes out of coverage of the ATS surveillance system, and position information shall be provided to all air traffic services units which may be able to give assistance to the aircraft. Transfer to adjacent sectors shall also be effected when appropriate.
    \note{If the pilot of an aircraft encountering a state of emergency has previously been directed by ATC to select a specific transponder code and/or an ADS-B emergency mode, that code/mode will normally be maintained unless, in special circumstances, the pilot has decided or has been advised otherwise. Where ATC has not requested a code or emergency mode to be set, the pilot will set the transponder to Mode A Code 7700 and/or the appropriate ADS-B emergency mode.}
    \item Whenever a general ADS-B emergency alert is observed on the situation display and there is no other indication of the particular nature of the emergency, the controller shall take the following action:
    
    \begin{enumalph}
        \item attempt to establish communication with the aircraft to verify the nature of the emergency; or
        \item if no response is received from the aircraft, the controller shall attempt to ascertain if the aircraft is able to receive transmissions from the air traffic control unit by requesting it to execute a specified manoeuvre which can be observed on the situation display.
    \end{enumalph}

    \note[1]{Some aircraft equipped with first generation ADS-B avionics have the capability to transmit a general emergency alert only, regardless of the code selected by the pilot.}
    \note[2]{Some aircraft equipped with first generation ADS-B avionics do not have the capability of squawking IDENT while the emergency and/or urgency mode is selected.}
\end{enumerate}

\subsubsection{Collision hazard information}

\begin{enumerate}
    \item When an identified controlled flight is observed to be on a conflicting path with an unknown aircraft deemed to constitute a collision hazard, the pilot of the controlled flight shall, whenever practicable:
    \begin{enumalph}
        \item be informed of the unknown aircraft, and if so requested by the controlled flight or if, in the opinion of the controller, the situation warrants, a course of avoiding action should be suggested; and
        \item be notified when the conflict no longer exists.
    \end{enumalph}

    \item When an identified IFR flight operating outside controlled airspace is observed to be on a conflicting path with another aircraft, the pilot should:
    \begin{enumalph}
        \item be informed as to the need for collision avoidance action to be initiated, and if so requested by the pilot or if, in the opinion of the controller, the situation warrants, a course of avoiding action should be suggested; and
        \item be notified when the conflict no longer exists.
    \end{enumalph}

    \item Information regarding traffic on a conflicting path should be given, whenever practicable, in the following form:
    \begin{enumalph}
        \item relative bearing of the conflicting traffic in terms of the 12-hour clock;
        \item distance from the conflicting traffic in nautical miles (or kilometres);
        \item direction in which the conflicting traffic appears to be proceeding;
        \item level and type of aircraft or, if unknown, relative speed of the conflicting traffic, e.g. slow or fast.
    \end{enumalph}

    \item Pressure-altitude-derived level information, even when unverified, should be used in the provision of collision hazard information because such information, particularly if available from an otherwise unknown aircraft (e.g. a VFR flight) and given to the pilot of a known aircraft, could facilitate the location of a collision hazard.
    \item When the pressure-altitude-derived level information has been verified, the information shall be passed to pilots in a clear and unambiguous manner. If the level information has not been verified, the accuracy of the information should be considered uncertain and the pilot shall be informed accordingly.
\end{enumerate}

% 8.8.3
% 8.8.4
% 8.8.5
% 8.8.6

\subsection[Use of ATS surveillance systems in the approach control service]{USE OF ATS SURVEILLANCE SYSTEMS IN THE \\ APPROACH CONTROL SERVICE}

\subsubsection{General provisions}

\begin{enumerate}
    \item ATS surveillance systems used in the provision of approach control service shall be appropriate to the functions and level of service to be provided.
    %References
    \item ATS surveillance systems used to monitor parallel ILS approaches shall meet the requirements for such operations specified in Chapter 6.
\end{enumerate}

\subsubsection{Functions}

The position indications presented on a situation display may be used to perform the following additional functions in the provision of approach control service:

\begin{enumalph}
    \item provide vectoring of arriving traffic on to pilot-interpreted final approach aids;
    \item provide flight path monitoring of parallel ILS approaches and instruct aircraft to take appropriate action in the event of possible or actual penetrations of the no transgression zone (NTZ);
    %References
    \note{See Chapter 6, Section 6.7.}
    \item provide vectoring of arriving traffic to a point from which a visual approach can be completed;
    \item provide vectoring of arriving traffic to a point from which a precision radar approach or a surveillance radar approach can be made;
    \item provide flight path monitoring of other pilot-interpreted approaches;
    \item in accordance with prescribed procedures, conduct:
    
    \begin{enumroman}
        \item surveillance radar approaches;
        \item precision radar (PAR) approaches; and
    \end{enumroman}

    \item provide separation between:
    
    \begin{enumroman}
        \item succeeding departing aircraft;
        \item succeeding arriving aircraft; and
        \item a departing aircraft and a succeeding arriving aircraft.
    \end{enumroman}
\end{enumalph}

\subsubsection[General approach control procedures using ATS surveillance systems]{General approach control procedures \\ using ATS surveillance systems}