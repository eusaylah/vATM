% ----- Chapter 9 ----- %
% FLIGHT INFORMATION SERVICE AND ALERTING SERVICE %
% Chapter 9 contains procedures applicable by air traffic services units providing flight information service and alerting service.

\chapterbegin

\section[Flight Information Service and Alerting Service]{FLIGHT INFORMATION SERVICE AND \\ ALERTING SERVICE}

\subsection[Flight information service]{FLIGHT INFORMATION SERVICE}

% 9.1.1
\stepcounter{subsubsection}

\subsubsection[Transfer of responsibility for the provision of flight information service]{Transfer of responsibility for the provision of \\ flight information service}

%References
The responsibility for the provision of flight information service to a flight normally passes from the appropriate ATS unit in an FIR to the appropriate ATS unit in the adjacent FIR at the time of crossing the common FIR boundary. However, when coordination is required in accordance with Chapter 10, 10.2, but communication facilities are inadequate, the former ATS unit shall, as far as practicable, continue to provide flight information service to the flight until it has established two-way communication with the appropriate ATS unit in the FIR it is entering.

\subsubsection{Transmission of information}

\begin{enumeratesc}
    \itemsc{Means of transmission}
    \begin{enumerate}
        \item \label{9.1.3.1.1} Except as provided in \ref{9.1.3.2.1}, information shall be disseminated to aircraft by one or more of the following means as determined by the appropriate ATS authority:
        \begin{enumalph}
            \item the preferred method of directed transmission on the initiative of the appropriate ATS unit to an aircraft, ensuring that receipt is acknowledged; or
            \item a general call, unacknowledged transmission to all aircraft concerned; or
            \item broadcast; or
            \item data link.
        \end{enumalph}

        \note{It should be recognized that in certain circumstances, e.g. during the last stages of a final approach, it may be impracticable for aircraft to acknowledge directed transmissions.}
        \item The use of general calls shall be limited to cases where it is necessary to disseminate essential information to several aircraft without delay, e.g. the sudden occurrence of hazards, a change of the runway-in-use, or the failure of a key approach and landing aid.
    \end{enumerate}

    \itemsc{Transmission of special air-reports, \\ SIGMET and AIRMET information}
    \begin{enumerate}
        \item \label{9.1.3.2.1} Appropriate SIGMET and AIRMET information, as well as special air-reports which have not been used for the preparation of a SIGMET, shall be disseminated to aircraft by one or more of the means specified in \ref{9.1.3.1.1} as determined on the basis of regional air navigation agreements. Special air-reports shall be disseminated to aircraft for a period of 60 minutes after their issuance.
        \item The special air-report, SIGMET and AIRMET information to be passed to aircraft on ground initiative should cover a portion of the route up to one hour's flying time ahead of the aircraft except when another period has been determined on the basis of regional air navigation agreements.
    \end{enumerate}

    \itemsc{Transmission of information concerning \\ volcanic activity}
    \begin{enumempty}
        \item Information concerning pre-eruption volcanic activity, volcanic eruptions and volcanic ash clouds (position of clouds and flight levels affected) shall be disseminated to aircraft by one or more of the means specified in \ref{9.1.3.1.1} as determined on the basis of regional air navigation agreements.
    \end{enumempty}

    \itemsc{Transmission of information concerning \\ radioactive materials and toxic chemical clouds}
    \begin{enumempty}
        \item Information on the release into the atmosphere of radioactive materials or toxic chemicals which could affect airspace within the area of responsibility of the ATS unit shall be transmitted to aircraft by one or more of the means specified in \ref{9.1.3.1.1}.
    \end{enumempty}

    \itemsc{Transmission of SPECI and amended TAF}
    \begin{enumerate}
        \item Special reports in the SPECI code form and amended TAF shall be transmitted on request and supplemented by:
        \begin{enumalph}
            \item directed transmission from the appropriate air traffic services unit of selected special reports and amended TAF for the departure, destination and its alternate aerodromes, as listed in the flight plan; or
            \item a general call on appropriate frequencies for the unacknowledged transmission to affected aircraft of selected special reports and amended TAF; or
            %\item continuous or frequent broadcast or the use of data link to make available current METAR and TAF in areas determined on the basis of regional air navigation agreements where traffic congestion dictates. VOLMET broadcasts and/or D-VOLMET should be used to serve this purpose (see Annex 11, 4.4).
        \end{enumalph}

        \item The passing of amended aerodrome forecasts to aircraft on the initiative of the appropriate air traffic services unit should be limited to that portion of the flight where the aircraft is within a specified time from the aerodrome of destination, such time being established on the basis of regional air navigation agreements.
    \end{enumerate}

    % 9.1.3.6
    \stepcounter{enumi}

    \itemsc{Transmission of information to supersonic aircraft}
    \begin{enumempty}
        \item The following information shall be available at appropriate ACCs or flight information centres for aerodromes determined on the basis of regional air navigation agreements and shall be transmitted on request to supersonic aircraft prior to commencement of deceleration/descent from supersonic cruise:
    \end{enumempty}
    \begin{enumalph}
        \item current meteorological reports and forecasts, except that where communications difficulties are encountered under conditions of poor propagation, the elements transmitted may be limited to:
        \begin{enumroman}
            \item mean surface wind, direction and speed (including gusts);
            \item visibility or runway visual range;
            \item amount and height of base of low clouds;
            \item other significant information;
            \item if appropriate, information regarding expected changes;
        \end{enumroman}

        \item operationally significant information on the status of facilities relating to the runway-in-use, including the precision approach category in the event that the lowest approach category promulgated for the runway is not available;
        \item sufficient information on the runway surface conditions to permit assessment of the runway braking action.
    \end{enumalph}
\end{enumeratesc}

\subsubsection{Air traffic advisory service}

\begin{enumeratesc}
    \itemsc{Objective and basic principles}
    \begin{enumerate}
        \item The objective of the air traffic advisory service is to make information on collision hazards more effective than it would be in the mere provision of flight information service. It may be provided to aircraft conducting IFR flights in advisory airspace or on advisory routes (Class F airspace). Such areas or routes will be specified by the State concerned.
        \item Taking into account the considerations detailed in 2.4 of Annex 11, air traffic advisory service should only be implemented where the air traffic services are inadequate for the provision of air traffic control, and the limited advice on collision hazards otherwise provided by flight information service will not meet the requirement. Where air traffic advisory service is implemented, this should be considered normally as a temporary measure only until such time as it can be replaced by air traffic control service.
        \item Air traffic advisory service does not afford the degree of safety and cannot assume the same responsibilities as air traffic control service in respect of the avoidance of collisions, since information regarding the disposition of traffic in the area concerned available to the unit providing air traffic advisory service may be incomplete. To make this quite clear, air traffic advisory service does not deliver ``clearances" but only ``advisory information" and it uses the word ``advise" or ``suggest" when a course of action is proposed to an aircraft.
        \note{See \ref{9.1.4.2.2}.}
    \end{enumerate}

    \itemsc{Aircraft}
    \begin{enumerate}[labelindent=0pt,itemsep=0.2cm]
        \itemscit{Aircraft using the air traffic advisory service} \label{9.1.4.2.1}
        
        \noindent IFR flights electing to use or required by the appropriate ATS authority on the basis of regional air navigation agreements to use the air traffic advisory service when operating within Class F airspace are expected to comply with the same procedures as those applying to controlled flights except that:

        \begin{enumalph}
            \item the flight plan and changes thereto are not subjected to a clearance, since the unit furnishing air traffic advisory service will only provide advice on the presence of essential traffic or suggestions as to a possible course of action;
            \note[1]{It is assumed that a pilot will not effect a change in the current flight plan until he or she has notified the intended change to the appropriate ATS unit and, if practicable, has received acknowledgement or relevant advice.}
            \note[2]{When a flight is operating or about to operate in a control area to continue eventually into an advisory area or along an advisory route, a clearance may be issued for the whole route, but the clearance as such, or revisions thereto, applies only to those portions of the flight conducted within control areas and control zones (3.7.4.4 of Annex 11). Advice or suggestions would be provided as necessary for the remaining portion of the route.}
            \item it is for the aircraft to decide whether or not it will comply with the advice or suggestion received and to inform the unit providing air traffic advisory service, without delay, of its decision;
            \item air-ground contacts shall be made with the air traffic services unit designated to provide air traffic advisory service within the advisory airspace or portion thereof.
            \note{See \ref{4}, \ref{4.4.2}, for procedures governing submission of a flight plan.}
        \end{enumalph}

        \itemscit{Aircraft not using the air traffic advisory service} \label{9.1.4.2.2}
        \begin{enumerate}
            \item Aircraft wishing to conduct IFR flights within advisory airspace, but not electing to use the air traffic advisory service, shall nevertheless submit a flight plan, and notify changes made thereto to the unit providing that service.
            \note{See \ref{4}, \ref{4.4.2}, for procedures governing submission of a flight plan.}
            \item IFR flights intending to cross an advisory route should do so as nearly as possible at an angle of 90 degrees to the direction of the route and at a level, appropriate to its track, selected from the tables of cruising levels prescribed for use by IFR flights operating outside controlled airspace.
        \end{enumerate}
    \end{enumerate}

    \itemsc{Air traffic services units}
    \begin{enumempty}
        \note{The efficiency of air traffic advisory service will depend largely on the procedures and practices in use. Its establishment in line with the organization, procedures and equipment of area control service, taking into account the basic differences of the two services, as indicated in \ref{9.1.4.2.1}, will help to ensure a high degree of efficiency and promote uniformity in the various provisions of air traffic advisory service. For example, exchange of information by the units concerned on the progress of an aircraft from one advisory area into an adjacent control area or terminal control area, and vice versa, will help to relieve pilots from repeating details of their flight plans already filed; also, use of standard air traffic control phraseology, preceded by the word \textup{``suggest"} or \textup{``advise"}, will facilitate the pilot's understanding of air traffic advisory service intelligence.}
    \end{enumempty}
    \begin{enumerate}
        \item An air traffic services unit providing air traffic advisory service shall:
        \begin{enumalph}
            \item \textit{advise} the aircraft to depart at the time specified and to cruise at the levels indicated in the flight plan if it does not foresee any conflict with other known traffic;
            \item \textit{suggest} to aircraft a course of action by which a potential hazard may be avoided, giving priority to an aircraft already in advisory airspace over other aircraft desiring to enter such advisory airspace; and
            \item \textit{pass} to aircraft traffic information comprising the same information as that prescribed for area control service.
        \end{enumalph}

        \item The criteria used as a basis for action under b) and c) above should be at least those laid down for aircraft operating in controlled airspace and should take into account the limitations inherent in the provision of air traffic advisory service, navigation facilities and air-ground communications prevailing in the region.
    \end{enumerate}
\end{enumeratesc}

\subsection[Alerting service]{ALERTING SERVICE}

\subsubsection{Aircraft}

\note{Whenever applied, the procedures for the provision of air traffic control service or air traffic advisory service take the place of the following procedures, except when relevant procedures do not call for more than hourly position reports, in which case the \textup{Operations normal} procedure applies.}

\begin{enumerate}
    \item When so required by the appropriate ATS authority to facilitate the provision of alerting and search and rescue services, an aircraft, prior to and when operating within or into designated areas or along designated routes, shall comply with the provisions detailed in Annex 2, Chapter 3, concerning the submission, completion, changing and closing of a flight plan.
    \item In addition to the above, aircraft equipped with suitable two-way radiocommunications shall report during the period twenty to forty minutes following the time of last contact, whatever the purpose of such contact, merely to indicate that the flight is progressing according to plan, such report to comprise identification of the aircraft and the words ``Operations normal" or the signal QRU.
    \item The ``Operations normal" message shall be transmitted air-ground to an appropriate air traffic services unit (e.g. normally to the aeronautical telecommunication station serving the air traffic services unit in charge of the FIR in which the aircraft is flying, otherwise to another aeronautical telecommunication station to be retransmitted as required to the air traffic services unit in charge of the FIR).

    % 9.2.1.4
\end{enumerate}

\subsubsection{Air traffic services units}

\begin{enumerate}
    \item When no report from an aircraft has been received within a reasonable period of time (which may be a specified interval prescribed on the basis of regional air navigation agreements) after a scheduled or expected reporting time, the ATS unit shall, within the stipulated period of thirty minutes, endeavour to obtain such report in order to be in a position to apply the provisions relevant to the ``Uncertainty Phase" (Annex 11, 5.2.1 refers) should circumstances warrant such application.
    \item \label{9.2.2.2} When alerting service is required in respect of a flight operated through more than one FIR or control area, and when the position of the aircraft is in doubt, responsibility for coordinating such service shall rest with the ATS unit of the FIR or control area:
    
    \begin{enumalph}
        \item within which the aircraft was flying at the time of last air-ground radio contact;
        \item that the aircraft was about to enter when last air-ground contact was established at or close to the boundary of two FIRs or control areas;
        \item within which the aircraft's intermediate stop or final destination point is located:
        \begin{enumroman}
            \item if the aircraft was not equipped with suitable two-way radiocommunication equipment; or
            \item was not under obligation to transmit position reports.
        \end{enumroman}
    \end{enumalph}

    \item The unit responsible for alerting service, in accordance with \ref{9.2.2.2}, shall:
    \begin{enumalph}
        \item notify units providing alerting service in other affected FIRs or control areas of the emergency phase or phases, in addition to notifying the rescue coordination centre associated with it;
        \item request those units to assist in the search for any useful information pertaining to the aircraft presumed to be in an emergency, by all appropriate means and especially those indicated in 5.3 of Annex 11 (Use of communication facilities);
        \item collect the information gathered during each phase of the emergency and, after verifying it as necessary, transmit it to the rescue coordination centre;
        \item announce the termination of the state of emergency as circumstances dictate.
    \end{enumalph}

    % 9.2.2.4
\end{enumerate}

\chapterend