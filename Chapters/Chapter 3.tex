% ----- Chapter 3 ----- %
% ATS SYSTEM CAPACITY AND AIR TRAFFIC FLOW MANAGEMENT %
% Chapter 3 contains provisions and procedures applicable to air traffic flow management.

\chapterbegin

\section[ATS System Capacity and Air Traffic Flow Management]{ATS SYSTEM CAPACITY AND \\ AIR TRAFFIC FLOW MANAGEMENT}

\subsection[Capacity management]{CAPACITY MANAGEMENT}

\subsubsection{General}

\begin{enumerate}
    \item The capacity of an ATS system depends on many factors, including the ATS route structure, the navigation accuracy of the aircraft using the airspace, weather-related factors, and controller workload. Every effort should be made to provide sufficient capacity to cater to both normal and peak traffic levels; however, in implementing any measures to increase capacity, the responsible ATS authority shall ensure, in accordance with the procedures specified in Chapter 2, that safety levels are not jeopardized.
    \item The number of aircraft provided with an ATC service shall not exceed that which can be safely handled by the ATC unit concerned under the prevailing circumstances. In order to define the maximum number of flights which can be safely accommodated, the appropriate ATS authority should assess and declare the ATC capacity for control areas, for control sectors within a control area and for aerodromes.
    \item ATC capacity should be expressed as the maximum number of aircraft which can be accepted over a given period of time within the airspace or at the aerodrome concerned.
    \note{The most appropriate measure of capacity is likely to be the sustainable hourly traffic flow. Such hourly capacities can, for example, be converted into daily, monthly or annual values.}
\end{enumerate}

\subsubsection{Capacity assessment}

In assessing capacity values, factors to be taken into account should include, \textit{inter alia}:

\begin{enumalph}
    \item the level and type of ATS provided;
    \item the structural complexity of the control area, the control sector or the aerodrome concerned;
    \item controller workload, including control and coordination tasks to be performed;
    \item the types of communications, navigation and surveillance systems in use, their degree of technical reliability and availability as well as the availability of backup systems and/or procedures;
    \item availability of ATC systems providing controller support and alert functions; and
    \item any other factor or element deemed relevant to controller workload.
\end{enumalph}

\note{Summaries of techniques which may be used to estimate control sector/position capacities are contained in the \textup{Air Traffic Services Planning Manual} (Doc 9426).}

\subsubsection{Regulation of ATC capacity and traffic volumes}

\begin{enumerate}
    \item Where traffic demand varies significantly on a daily or periodic basis, facilities and procedures should be implemented to vary the number of operational sectors or working positions to meet the prevailing and anticipated demand. Applicable procedures should be contained in local instructions.
    \item In case of particular events which have a negative impact on the declared capacity of an airspace or aerodrome, the capacity of the airspace or aerodrome concerned shall be reduced accordingly for the required time period. Whenever possible, the capacity pertaining to such events should be predetermined.
    \item To ensure that safety is not compromised whenever the traffic demand in an airspace or at an aerodrome is forecast to exceed the available ATC capacity, measures shall be implemented to regulate traffic volumes accordingly.
\end{enumerate}

% 3.1.4
\stepcounter{subsubsection}

\subsubsection{Flexible use of airspace}

\begin{enumerate}
    \item The appropriate authorities should, through the establishment of agreements and procedures, make provision for the flexible use of all airspace in order to increase airspace capacity and to improve the efficiency and flexibility of aircraft operations. When applicable, such agreements and procedures should be established on the basis of a regional air navigation agreement.
    \item Agreements and procedures providing for a flexible use of airspace should specify, \textit{inter alia}:
    
    \begin{enumalph}
        \item the horizontal and vertical limits of the airspace concerned;
        \item the classification of any airspace made available for use by civil air traffic;
        \item units or authorities responsible for transfer of the airspace;
        \item conditions for transfer of the airspace to the ATC unit concerned;
        \item conditions for transfer of the airspace from the ATC unit concerned;
        \item periods of availability of the airspace;
        \item any limitations on the use of the airspace concerned; and
        \item any other relevant procedures or information.
    \end{enumalph}
\end{enumerate}

\subsection[Air traffic flow management]{AIR TRAFFIC FLOW MANAGEMENT}

\subsubsection{General}

\begin{enumerate}
    \item An air traffic flow management (ATFM) service shall be implemented for airspace where traffic demand at times exceeds the defined ATC capacity.
    \item ATFM should be implemented on the basis of a regional air navigation agreement or, when appropriate, as a multilateral agreement.
    \item The ATFM service within a region or other defined area, should be developed and implemented as a centralized ATFM organization, supported by flow management positions established at each area control centre (ACC) within the region or area of applicability.
    \item Certain flights may be exempt from ATFM measures, or be given priority over other flights.
    \item Detailed procedures governing the provision of the ATFM measures, and service within a region or area should be prescribed in a regional ATFM manual or handbook.
\end{enumerate}

% 3.2.2 3.2.3 3.2.4
\addtocounter{subsubsection}{3}

\subsubsection{Tactical operations}

\begin{enumerate}
    \item Tactical ATFM operations should consist of:
    \begin{enumalph}
        \item executing the agreed tactical measures in order to provide a reduced and even flow of traffic where demand would otherwise have exceeded capacity;
        \item monitoring the evolution of the air traffic situation to ensure that the ATFM measures applied are having the desired effect and to take or initiate remedial action when long delays are reported, including re-routing of traffic and flight level allocation, in order to utilize the available ATC capacity to the maximum extent.
    \end{enumalph}

    \item When the traffic demand exceeds, or is foreseen to exceed, the capacity of a particular sector or aerodrome, the responsible ATC unit shall advise the responsible ATFM unit, where such a unit is established, and other ATC units concerned. Flight crews of aircraft planned to fly in the affected area and operators should be advised, as soon as practicable, of the delays expected or the restrictions which will be applied.
\end{enumerate}

% 3.2.6

\chapterend