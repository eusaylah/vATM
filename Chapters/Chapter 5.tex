% ----- Chapter 5 ----- %
% SEPARATION METHODS AND MINIMA %
% Chapter 5 contains provisions and procedures applicable to the separation of aircraft.

\chapterbegin

\section[Separation Methods and Minima]{SEPARATION METHODS AND MINIMA}

\subsection[Introduction]{INTRODUCTION}

%References
\note[1]{With the exceptions stated below, Chapter 5 contains procedures and procedural separation minima for use in the separation of aircraft in the en-route phase as well as aircraft in the arrival and departure phases of flight.}
\note[2]{Procedures and separation minima applicable to approaches to parallel runways are contained in Chapter 6. Procedures and separation minima applicable in the provision of aerodrome control service are contained in Chapter 7 and procedures and separation minima applicable to the use of ATS surveillance systems are contained in Chapter 8.}
\note[3]{Attention is drawn to the use of strategic lateral offset procedures (SLOP) described in Chapter 16, 16.5.}
\note[4]{Procedures applicable to data link initiation capability (DLIC) are contained in Chapter 4. Procedures applicable to automatic dependent surveillance — contract (ADS-C) are contained in Chapter 13. Procedures applicable to controller-pilot data link communications (CPDLC) are contained in Chapter 14.}

\subsection[Provisions for the separation of controlled traffic]{PROVISIONS FOR THE SEPARATION OF CONTROLLED TRAFFIC}

\subsubsection{General}

\begin{enumerate}
    %References
    \item Vertical or horizontal separation shall be provided:
    \begin{enumalph}
        \item between all flights in Class A and B airspaces;
        \item between IFR flights in Class C, D and E airspaces;
        \item between IFR flights and VFR flights in Class C airspace;
        \item between IFR flights and special VFR flights; and
        \item between special VFR flights, when so prescribed by the appropriate ATS authority;
    \end{enumalph}

    \noindent except, for the cases under b) above in airspace Classes D and E, during the hours of daylight when flights have been cleared to climb or descend subject to maintaining own separation and remaining in visual meteorological conditions. Conditions applicable to the use of this procedure are contained in Section 5.9.
    \item No clearance shall be given to execute any manoeuvre that would reduce the spacing between two aircraft to less than the separation minimum applicable in the circumstances.
    \item Larger separations than the specified minima should be applied whenever exceptional circumstances such as navigational difficulties call for extra precautions. This should be done with due regard to all relevant factors so as to avoid impeding the flow of air traffic by the application of excessive separations.
    \item Where the type of separation or minimum used to separate two aircraft cannot be maintained, another type of separation or another minimum shall be established prior to the time when the current separation minimum would be infringed.
\end{enumerate}

\subsubsection{Degraded aircraft performance}

Whenever, as a result of failure or degradation of navigation, communications, altimetry, flight control or other systems, aircraft performance is degraded below the level required for the airspace in which it is operating, the flight crew shall advise the ATC unit concerned without delay. Where the failure or degradation affects the separation minimum currently being employed, the controller shall take action to establish another appropriate type of separation or separation minimum.

\subsection[Vertical separation]{VERTICAL SEPARATION}

\subsubsection{Vertical separation application}

%References
Vertical separation is obtained by requiring aircraft using prescribed altimeter setting procedures to operate at different levels expressed in terms of flight levels or altitudes in accordance with the provisions in Chapter 4, Section 4.10.

\subsubsection{Vertical separation minimum}

The vertical separation minimum (VSM) shall be:

\begin{enumalph}
    \item a nominal 1 000 ft below FL 290 and a nominal 2 000 ft at or above this level, except as provided for in \ref{5.3.2b} below; and
    \item \label{5.3.2b} within designated airspace, subject to a regional air navigation agreement: a nominal 1 000 ft below FL 410 or a higher level where so prescribed for use under specified conditions, and a nominal 2 000 ft at or above this level.
\end{enumalph}

\note{Guidance material relating to vertical separation is contained in the \textup{Manual on a 300 m (1 000 ft) Vertical
Separation Minimum Between FL 290 and FL 410 Inclusive} (Doc 9574).}

\subsubsection{Assignment of cruising levels for controlled flights}

\begin{enumerate}
    \item Except when traffic conditions and coordination procedures permit authorization of cruise climb, an ATC unit shall normally authorize only one level for an aircraft beyond its control area, i.e. that level at which the aircraft will enter the next control area whether contiguous or not. It is the responsibility of the accepting ATC unit to issue clearance for further climb as appropriate. When relevant, aircraft will be advised to request en route any cruising level changes desired.
    \item Aircraft authorized to employ cruise climb techniques shall be cleared to operate between two levels or above a level.
    \item If it is necessary to change the cruising level of an aircraft operating along an established ATS route extending partly within and partly outside controlled airspace and where the respective series of cruising levels are not identical, the change shall, whenever possible, be effected within controlled airspace.
    \item When an aircraft has been cleared into a control area at a cruising level which is below the established minimum cruising level for a subsequent portion of the route, the ATC unit responsible for the area should issue a revised clearance to the aircraft even though the pilot has not requested the necessary cruising level change.
    \item An aircraft may be cleared to change cruising level at a specified time, place or rate.
    %References
    \note{See 5.3.4.1.1 concerning procedures for vertical speed control.}
    \item In so far as practicable, cruising levels of aircraft flying to the same destination shall be assigned in a manner that will be correct for an approach sequence at destination.
    \item An aircraft at a cruising level shall normally have priority over other aircraft requesting that cruising level. When two or more aircraft are at the same cruising level, the preceding aircraft shall normally have priority.
    \item The cruising levels, or, in the case of cruise climb, the range of levels, to be assigned to controlled flights shall be selected from those allocated to IFR flights in:
    
    \begin{enumalph}
        \item the tables of cruising levels in Appendix 3 of Annex 2; or
        \item a modified table of cruising levels, when so prescribed in accordance with Appendix 3 of Annex 2 for flights above FL 410;
    \end{enumalph}

    \noindent except that the correlation of levels to track as prescribed therein shall not apply whenever otherwise indicated in air traffic control clearances or specified by the appropriate ATS authority in AIPs.
\end{enumerate}

\subsubsection{Vertical separation during climb or descent}

\begin{enumerate}
    \item An aircraft may be cleared to a level previously occupied by another aircraft after the latter has reported vacating it, except when:
    \begin{enumalph}
        \item severe turbulence is known to exist;
        \item the higher aircraft is effecting a cruise climb; or
        \item the difference in aircraft performance is such that less than the applicable separation minimum may result;
    \end{enumalph}

    \noindent in which case such clearance shall be withheld until the aircraft vacating the level has reported at or passing another level separated by the required minimum.

    \begin{enumerate}
        \item When the aircraft concerned are entering or established in the same holding pattern, consideration shall be given to aircraft descending at markedly different rates and, if necessary, additional measures such as specifying a maximum descent rate for the higher aircraft and a minimum descent rate for the lower aircraft should be applied to ensure that the required separation is maintained.
    \end{enumerate}

    \item Pilots in direct communication with each other may, with their concurrence, be cleared to maintain a specified vertical separation between their aircraft during ascent or descent.
\end{enumerate}

\subsection[Horizontal separation]{HORIZONTAL SEPARATION}

%References
\begin{noteev}
    \note[1]{Nothing in the provisions detailed in Sections 5.4.1 and 5.4.2 hereunder precludes a State from establishing:}
    \begin{enumalph}
        \item other minima for use in circumstances not prescribed; or
        \item additional conditions to those prescribed for the use of a given minimum;
    \end{enumalph}
    \noindent provided that the level of safety inherent in the provisions detailed in Sections 5.4.1 and 5.4.2 hereunder is at all times assured.

    \note[2]{Details on track spacing between parallel routes are provided in Annex 11, Attachments A and B.}
    \note[3]{Attention is drawn to the following guidance material:}
    \begin{enumalph}
        \item \textup{Air Traffic Services Planning Manual} (Doc 9426);
        \item \textup{Manual on Airspace Planning Methodology for the Determination of Separation Minima} (Doc 9689); and
        \item \textup{Performance-based Navigation (PBN) Manual} (Doc 9613).
    \end{enumalph}

    \note[4]{Provisions concerning reductions in separation minima are contained in Section 5.11 and in Chapter 2,
    ATS Safety Management.}
\end{noteev}

\subsubsection{Lateral separation}

\begin{enumeratesc}
    \itemsc{Lateral separation application}
    \begin{enumerate}
        \item Lateral separation shall be applied so that the distance between those portions of the intended routes for which the aircraft are to be laterally separated is never less than an established distance to account for navigational inaccuracies plus a specified buffer. This buffer shall be determined by the appropriate authority and included in the lateral separation minima as an integral part thereof.
        %References
        \note{In the minima specified in 5.4.1.2 an appropriate buffer has already been included.}
        \item Lateral separation of aircraft is obtained by requiring operation on different routes or in different geographical locations as determined by visual observation, by the use of navigation aids or by the use of area navigation (RNAV) equipment.
        \item When information is received indicating navigation equipment failure or deterioration below the navigation performance requirements, ATC shall then, as required, apply alternative separation methods or minima.
        %References
        \item When an aircraft turns onto an ATS route via a flyover waypoint, a separation other than the normally prescribed lateral separation shall be applied for that portion of the flight between the flyover waypoint where the turn is executed and the next waypoint (see Figures 5-1 and 5-2).
        \note[1]{For flyover waypoints aircraft are required to first fly over the waypoint before executing the turn. After the turn the aircraft may either navigate to join the route immediately after the turn or navigate to the next defined waypoint before re-joining the route. This will require additional lateral separation on the overflown side of the turn.}
        \note[2]{This does not apply to ATS routes that have turns using fly-by waypoints.}
        %References
        \note[3]{An example of a prescribed lateral separation minima based on a specific navigation performance can be found in 5.4.1.2.1.6.}
    \end{enumerate}

    %Figure 5-1 and 5-2

    \itemsc{Lateral separation criteria and minima}
    \begin{enumerate}
        \item Means by which lateral separation may be applied include the following:
        \begin{enumerate}
            %References
            \item \textit{By reference to the same or different geographic locations.} By position reports which positively indicate the aircraft are over different geographic locations as determined visually or by reference to a navigation aid (see Figure 5-3).
            \item \textit{By use of NDB, VOR or GNSS on intersecting tracks or ATS routes.} By requiring aircraft to fly on specified tracks which are separated by a minimum amount appropriate to the navigation aid employed. Lateral separation between two aircraft exists when:
            \begin{enumalph}
                %References
                \item \textit{VOR:} both aircraft are established on radials diverging by at least 15 degrees and at least one aircraft is at a distance of 15 NM or more from the facility (see Figure 5-4);
                \item \textit{NDB:} both aircraft are established on tracks to or from the NDB which are diverging by at least 30 degrees and at least one aircraft is at a distance of 15 NM or more from the facility (see Figure 5-5);
                \item \textit{GNSS/GNSS:} each aircraft is confirmed to be established on a track with zero offset between two waypoints and at least one aircraft is at a minimum distance from a common point as specified in Table 5-1; or
                \item \textit{VOR/GNSS:} the aircraft using VOR is established on a radial to or from the VOR and the other aircraft using GNSS is confirmed to be established on a track with zero offset between two waypoints and at least one aircraft is at a minimum distance from a common point as specified in Table 5-1.
            \end{enumalph}

            %Table 5-1
            \vspace{0.5\parsep}
            \begin{table}[h]
                \centering
                \caption{Lateral separation for aircraft flying VOR and GNSS}
                \label{table:5-1}
                \begin{tabularx}{\textwidth}{|Sc|X|X|}
                    \hline
                    & \multicolumn{2}{Sc|}{\itshape \makecell{Aircraft 1: VOR or GNSS \\ Aircraft 2: GNSS}} \\
                    \hline
                    \makecell{Angular difference between tracks \\ measured at the common point \\ (degrees)} & \makecell{FL010 -- FL190 \\ Distance from a common point} & \makecell{FL200 -- FL600 \\ Distance from a common point} \\
                    \hline
                    \makecell{15 -- 135} & \makecell{15 NM} & \makecell{23 NM} \\
                    \hline
                    \multicolumn{3}{|Sl|}{\makecell[l]{The distances in the table are ground distances. \\ States must take into account the distance (slant range) from the source of a DME signal to the \\ receiving antenna when DME is being utilized to provide range information.}} \\
                    \hline
                \end{tabularx}
            \end{table}

            %References
            \note[1]{The values in Table 5-1 are from a larger table of values derived by collision risk analysis. The source table for separation of aircraft navigating by means of GNSS and VOR is contained in Circular 322, \textup{Guidelines for the Implementation of GNSS Lateral Separation Minima based on VOR Separation Minima}. States may refer to Circular 322 for greater detail and other angular differences and separation distances.}
            \note[2]{The values in Table 5-1 have accounted for distances from the common point encompassed by the theoretical turn area for fly-by turns as specified in the \textup{Minimum Aviation System Performance Standard: Required Navigation Performance for Air Navigation} (ED-75B/DO-236B), section 3.2.5.4, and fixed radius transition turns as defined in the \textup{Performance-based Navigation (PBN) Manual} (Doc 9613).}
            \note[3]{Guidance material for the implementation of GNSS lateral separation is contained in Circular 322, \textup{Guidelines for the Implementation of GNSS Lateral Separation Minima Based on VOR Separation Minima}.}

            %Figure 5-3 5-4 5-5

            \begin{enumerate}
                %References
                \item When aircraft are operating on tracks which are separated by considerably more than the minimum in 5.4.1.2.1.2 a) and b), States may reduce the distance at which lateral separation is achieved.
                \item Before applying GNSS-based track separation, the controller shall confirm the following:
                \begin{enumalph}
                    \item ensure that the aircraft is navigating using GNSS; and
                    \item in airspace where strategic lateral offsets are authorized, that a lateral offset is not being applied.
                \end{enumalph}
                \item In order to minimize the possibility of operational errors, waypoints contained in the navigation database or uplinked to the aircraft flight management system should be used in lieu of manually entered waypoints, when applying GNSS-based track separation. In the event that it is operationally restrictive to use waypoints contained in the navigation database, the use of waypoints that require manual entry by pilots should be limited to a half or whole degree of latitude and longitude.
                \item GNSS-based track separation shall not be applied in cases of pilot-reported receiver autonomous integrity monitoring (RAIM) outages.
                \note{For the purpose of applying GNSS-based lateral separation minima, distance and track information derived from an integrated navigation system incorporating GNSS input is regarded as equivalent to GNSS distance and track.}
                \item GNSS receivers used for applying separation shall meet the requirements in Annex 10, Volume I, and be indicated in the flight plan.
            \end{enumerate}
            
            \item \textit{By use of different navigation aids or methods.} Lateral separation between aircraft using different navigation aids, or when one aircraft is using RNAV equipment, shall be established by ensuring that the derived protected airspaces for the navigation aid(s) or RNP do not overlap.
            \item \textit{Lateral separation of aircraft on published instrument flight procedures for arrivals and departures.}
            
            \begin{enumerate}
                \item Lateral separation of departing and/or arriving aircraft, using instrument flight procedures, will exist:
                \begin{enumalph}
                    \item where the distance between any combination of RNAV 1 with RNAV 1, or RNP 1, RNP APCH or RNP AR APCH tracks is not less than 7 NM; or
                    \item where the distance between any combination of RNP 1, RNP APCH or RNP AR APCH tracks is not less than 5 NM; or
                    \item where the protected areas of tracks designed using obstacle clearance criteria do not overlap and provided operational error is considered.
                \end{enumalph}

                %References
                \note[1]{Distance values contained in a) and b) above were determined by collision risk analysis using multiple navigation specifications. Information on this analysis is contained in Circular 324, \textup{Guidelines for Lateral Separation of Arriving and Departing Aircraft on Published Adjacent Instrument Flight Procedures}.}
                \note[2]{Circular 324 also contains information on separation of arrival and departure tracks using non-overlapping protected areas based on obstacle clearance criteria, as provided for in the \textup{Procedures for Air Navigation Services -- Aircraft Operations}, Volume II -- \textup{Construction of Visual and Instrument Flight Procedures} (PANS-OPS, Doc 8168).}
                \note[3]{Provisions concerning reductions in separation minima are contained in Chapter 2, ATS Safety Management, and Chapter 5, Separation Methods and Minima, Section 5.11.}
                \note[4]{Guidance concerning the navigation specifications is contained in the \textup{Performance-based Navigation (PBN) Manual} (Doc 9613).}
            \end{enumerate}

            \item \textit{RNAV operations where RNP is specified on parallel tracks or ATS routes.} Within designated airspace or on designated routes, where RNP is specified, lateral separation between RNAV-equipped aircraft may be obtained by requiring aircraft to be established on the centre lines of parallel tracks or ATS routes spaced at a distance which ensures that the protected airspace of the tracks or ATS routes does not overlap.
            \note{The spacing between parallel tracks or between parallel ATS route centre lines for which an RNP type is required will be dependent upon the relevant RNP type specified. Guidance material related to the spacing between tracks or ATS routes based on RNP type is contained in Annex 11, Attachment B.}

            \item \textit{Lateral separation of aircraft on parallel or non-intersecting tracks or ATS routes.} Within designated airspace or on designated routes, lateral separation between aircraft operating on parallel or non-intersecting tracks or ATS routes shall be established in accordance with the following:
            \begin{enumalph}
                \item for a minimum spacing between tracks of 50 NM a navigational performance of RNAV 10 (RNP 10), RNP 4 or RNP 2 shall be prescribed;
                \item for a minimum spacing between tracks of 23 NM a navigational performance of RNP 4 or RNP 2 shall be prescribed. The communication system shall satisfy required communication performance 240 (RCP 240) and the surveillance system shall satisfy required surveillance performance 180 (RSP 180). Conformance monitoring shall be ensured by establishing an ADS-C event contract specifying a lateral deviation change event with a maximum of 5 NM threshold and a waypoint change event;
                \item for a minimum spacing between tracks of 15 NM a navigational performance of RNP 2 or a GNSS equipage shall be prescribed. Direct controller-pilot VHF voice communication shall be maintained while such separation is applied;
                \item \label{5.4.1.2.1.6d} for a minimum spacing between tracks of 7 NM, applied while one aircraft climbs/descends through the level of another aircraft, a navigational performance of RNP 2 or a GNSS equipage shall be prescribed. Direct controller-pilot VHF voice communication shall be maintained while such separation is applied; and
                \item for a minimum spacing between tracks of 20 NM, applied while one aircraft climbs/descends through the level of another aircraft whilst using other types of communication than specified in \ref{5.4.1.2.1.6d} above, a navigational performance of RNP 2 or a GNSS equipage shall be prescribed.
            \end{enumalph}

            \note[1]{Guidance material for the implementation of the navigation capability supporting 50 NM, 23 NM, 20 NM, 15 NM and 7 NM lateral separation minima is contained in the \textup{Performance-based Navigation (PBN) Manual} (Doc 9613). Guidance material for the implementation of the 50 NM, 23 NM, 20 NM, 15 NM and 7 NM lateral separation minima is contained in Circular 341, \textup{Guidelines for the Implementation of Lateral Separation Minima}.}
            \note[2]{Guidance material for implementation of communication and surveillance capability supporting 50 NM and 23 NM lateral separation minima is contained in the \textup{Performance-based Communication and Surveillance (PBCS) Manual} (Doc 9869) and the \textup{Global Operational Data Link (GOLD) Manual} (Doc 10037).}
            %References
            \note[3]{See Appendix 2, ITEM 10: EQUIPMENT AND CAPABILITIES, in relation to the GNSS prescribed in c), d) and e) above.}

            \item \textit{Lateral separation of aircraft on intersecting tracks or ATS routes.} Lateral separation between aircraft operating on intersecting tracks or ATS routes shall be established in accordance with the following.
            \begin{enumalph}
                %References
                \item an aircraft converging with the track of another aircraft is laterally separated until it reaches a lateral separation point that is located a specified distance measured perpendicularly from the track of the other aircraft (see Figure 5-6); and
                \item an aircraft diverging from the track of another aircraft is laterally separated after passing a lateral separation point that is located a specified distance measured perpendicularly from the track of the other aircraft (see Figure 5-6).
            \end{enumalph}

            \noindent This type of separation may be used for tracks that intersect at any angles using the values for lateral separation points specified below:
    
            \begin{tablecenter}[2cm]{ScSc}
                \toprule
                \textit{Navigation} & \textit{Separation} \\
                \midrule
                RNAV 10 (RNP 10) & 50 NM \\
                RNP 4 & 23 NM \\
                RNP 2 & 15 NM \\
                \bottomrule
            \end{tablecenter}

            \item When applying the 15 NM separation minima specified in the table above, a GNSS, as indicated in the flight plan by the letter G meets the specified navigation performance.
            \note{Guidance material for the implementation of the navigation capability supporting 50 NM, 23 NM, and 15 NM lateral separation minima is contained in the \textup{Performance-based Navigation (PBN) Manual} (Doc 9613). Supporting information for the implementation of the 50 NM, 23 NM and 15 NM lateral separation minima is contained in Circular 341, \textup{Guidelines for the Implementation of Lateral Separation Minima}.}

            %Figure 5-6

            \item \textit{Transitioning into airspace where a greater lateral separation minimum applies.} Lateral separation will exist when aircraft are established on specified tracks which:
            \begin{enumalph}
                \item are separated by an appropriate minimum; and
                \item diverge by at least 15 degrees until the applicable lateral separation minimum is established;
            \end{enumalph}

            \noindent providing that it is possible to ensure, by means approved by the appropriate ATS authority, that aircraft have the navigation capability necessary to ensure accurate track guidance.
        \end{enumerate}
    \end{enumerate}
\end{enumeratesc}

\subsubsection{Longitudinal separation}

\begin{enumeratesc}
    \itemsc{Longitudinal separation application}
    \begin{enumerate}
        \item Longitudinal separation shall be applied so that the spacing between the estimated positions of the aircraft being separated is never less than a prescribed minimum. Longitudinal separation between aircraft following the same or diverging tracks may be maintained by application of speed control, including the Mach number technique. When applicable, use of the Mach number technique shall be prescribed on the basis of a regional air navigation agreement.
        \note[1]{Attention is drawn to the guidance material contained in the \textup{Air Traffic Services Planning Manual} (Doc 9426) regarding the application of the Mach number technique to separation of subsonic aircraft.}
        \note[2]{The Mach number technique is applied using true Mach number.}
        \item In applying a time- or distance-based longitudinal separation minimum between aircraft following the same track, care shall be exercised to ensure that the separation minimum will not be infringed whenever the following aircraft is maintaining a higher airspeed than the preceding aircraft. When aircraft are expected to reach minimum separation, speed control shall be applied to ensure that the required separation minimum is maintained.
        \item Longitudinal separation may be established by requiring aircraft to depart at a specified time, to arrive over a geographical location at a specified time, or to hold over a geographical location until a specified time.
        \item Longitudinal separation between supersonic aircraft during the transonic acceleration and supersonic phases of flight should normally be established by appropriate timing of the start of transonic acceleration rather than by the imposition of speed restrictions in supersonic flight.
        \item For the purpose of application of longitudinal separation, the terms \textit{same track}, \textit{reciprocal tracks} and \textit{crossing tracks} shall have the following meanings:
        
        \begin{enumalph}
            %References
            \item Same track (see Figure 5-7):
            \par\noindent same direction tracks and intersecting tracks or portions thereof, the angular difference of which is less than 45 degrees or more than 315 degrees, and whose protected airspaces overlap.
            \item Reciprocal tracks (see Figure 5-8):
            \par\noindent opposite tracks and intersecting tracks or portions thereof, the angular difference of which is more than 135 degrees but less than 225 degrees, and whose protected airspaces overlap.
            \item Crossing tracks (see Figure 5-9):
            \par\noindent intersecting tracks or portions thereof other than those specified in a) and b) above.
        \end{enumalph}

        %References
        \item Time-based separation applied in accordance with 5.4.2.2 and 5.4.2.4 may be based on position
        information and estimates derived from voice reports, CPDLC or ADS-C.
    \end{enumerate}
\end{enumeratesc}