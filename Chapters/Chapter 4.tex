% ----- Chapter 4 ----- %
% GENERAL PROVISIONS FOR AIR TRAFFIC SERVICES %
% Chapter 4 contains general provisions and procedures applicable to the air traffic services.

\chapterbegin

\section[General Provisions for Air Traffic Services]{GENERAL PROVISIONS FOR AIR TRAFFIC SERVICES} \label{4}

\subsection[Responsibility for the provision of air traffic control service]{RESPONSIBILITY FOR THE PROVISION OF \\ AIR TRAFFIC CONTROL SERVICE}

\subsubsection{Area control service}

Area control service shall be provided:
\begin{enumalph}
    \item by an area control centre (ACC); or
    \item by the unit providing approach control service in a control zone or in a control area of limited extent which is designated primarily for the provision of approach control service, when no ACC is established.
\end{enumalph}

\subsubsection{Approach control service}

Approach control service shall be provided:
\begin{enumalph}
    \item by an aerodrome control tower or an ACC, when it is necessary or desirable to combine under the responsibility of one unit the functions of the approach control service and those of the aerodrome control service or the area control service; or
    \item by an approach control unit, when it is necessary or desirable to establish a separate unit.
\end{enumalph}

\note{Approach control service may be provided by a unit collocated with an ACC, or by a control sector within an ACC.}

\subsubsection{Approach control service}

Aerodrome control service shall be provided by an aerodrome control tower.

\subsection[Responsibility for the provision of flight information service and alerting service]{RESPONSIBILITY FOR THE PROVISION OF FLIGHT \\ INFORMATION SERVICE AND ALERTING SERVICE}

Flight information service and alerting service shall be provided as follows:
\begin{enumalph}
    \item \textit{within a flight information region (FIR):} by a flight information centre, unless the responsibility for providing such services is assigned to an air traffic control unit having adequate facilities for the exercise of such responsibilities;
    \item \textit{within controlled airspace and at controlled aerodromes:} by the relevant air traffic control units.
\end{enumalph}

\subsection[Division of responsibility for control between air traffic control units]{DIVISION OF RESPONSIBILITY FOR CONTROL \\ BETWEEN AIR TRAFFIC CONTROL UNITS}

\subsubsection{General}

The appropriate ATS authority shall designate the area of responsibility for each air traffic control (ATC) unit and, when applicable, for individual control sectors within an ATC unit. Where there is more than one ATC working position within a unit or sector, the duties and responsibilities of the individual working positions shall be defined.

\subsubsection[Between a unit providing aerodrome control service and a unit providing approach control service]{Between a unit providing aerodrome control service \\ and a unit providing approach control service}

\begin{enumerate}
    \item Except for flights which are provided aerodrome control service only, the control of arriving and departing controlled flights shall be divided between units providing aerodrome control service and units providing approach control service as follows:
    
    \begin{enumerate}
        \item \textit{Arriving aircraft.} Control of an arriving aircraft shall be transferred from the unit providing approach control service to the unit providing aerodrome control service when the aircraft:
        
        \begin{enumalph}
            \item is in the vicinity of the aerodrome, and
            \begin{enumarab}
                \item it is considered that approach and landing will be completed in visual reference to the ground, or
                \item has reached uninterrupted visual meteorological conditions, or
            \end{enumarab}
            \item is at a prescribed point or level, or
            \item has landed,
        \end{enumalph}
        
        \noindent as specified in letters of agreement or ATS unit instructions.
        
        \item \label{4.3.2.1.2} Transfer of communications to the aerodrome controller should be effected at such a point, level or time that clearance to land or alternative instructions, as well as information on essential local traffic, can be issued in a timely manner.
        \note{Even though there is an approach control unit, control of certain flights may be transferred directly from an ACC to an aerodrome control tower and vice versa, by prior arrangement between the units concerned for the relevant part of approach control service to be provided by the ACC or the aerodrome control tower, as applicable.} 
        
        \item \textit{Departing aircraft.} Control of a departing aircraft shall be transferred from the unit providing aerodrome control service to the unit providing approach control service:

        \begin{enumalph}
            \item when visual meteorological conditions prevail in the vicinity of the aerodrome:

            \begin{enumarab}
                \item prior to the time the aircraft leaves the vicinity of the aerodrome,
                \item prior to the aircraft entering instrument meteorological conditions, or
                \item when the aircraft is at a prescribed point or level,
            \end{enumarab}
        \end{enumalph}
            
        \noindent as specified in letters of agreement or ATS unit instructions;

        \begin{enumalph}[resume*]
            \item when instrument meteorological conditions prevail at the aerodrome:

            \begin{enumarab}
                \item immediately after the aircraft is airborne, or
                \item when the aircraft is at a prescribed point or level,
            \end{enumarab}
        \end{enumalph}
        
        \noindent as specified in letters of agreement or local instructions.

        %References
        \note{See Note following \ref{4.3.2.1.2}.}
    \end{enumerate}
\end{enumerate}

\subsubsection[Between a unit providing approach control service and a unit providing area control service]{Between a unit providing approach control service \\ and a unit providing area control service}

\begin{enumerate}
    \item When area control service and approach control service are not provided by the same air traffic control unit, responsibility for controlled flights shall rest with the unit providing area control service except that a unit providing approach control service shall be responsible for the control of:

    \begin{enumalph}
        \item arriving aircraft that have been released to it by the ACC;
        \item departing aircraft until such aircraft are released to the ACC.
    \end{enumalph}

    \item A unit providing approach control service shall assume control of arriving aircraft, provided such aircraft have been released to it, upon arrival of the aircraft at the point, level or time agreed for transfer of control, and shall maintain control during approach to the aerodrome.
\end{enumerate}

\subsubsection{Between two units providing area control service}

The responsibility for the control of an aircraft shall be transferred from a unit providing area control service in a control area to the unit providing area control service in an adjacent control area at the time of crossing the common control area boundary as estimated by the ACC having control of the aircraft or at such other point, level or time as has been agreed between the two units.

\subsubsection{Between control sectors/positions within the same air traffic control unit}

The responsibility for the control of an aircraft shall be transferred from one control sector/position to another control sector/position within the same ATC unit at a point, level or time, as specified in local instructions.

\subsection[Flight plan]{FLIGHT PLAN}

\subsubsection{Flight plan form}

%Numbering
\begin{enumerate}
    \item An operator shall, prior to departure:
    \begin{enumalph}
        \item ensure that, where the flight is intended to operate on a route or in an area where a navigation specification is prescribed, it has an appropriate RNP approval, and that all conditions applying to that approval will be satisfied;
        \item ensure that, where the flight is intended to operate in reduced vertical separation minimum (RVSM) airspace, it has the required RVSM approval;
        \item ensure that, where the flight is intended to operate where an RCP specification is prescribed, it has an appropriate approval, and that all conditions applying to that approval will be satisfied.
        \item ensure that, where the flight is intended to operate where an RSP specification is prescribed, it has an appropriate RSP approval, and that all conditions applying to that approval will be satisfied.
    \end{enumalph}
\end{enumerate}

%Numbering
\subsubsection{Submission of a flight plan}

\begin{enumeratesc}
    \itemsc{Prior to departure}
    \begin{enumempty}[labelindent=\parindent]
        %VATSIM
        \item \note{On VATSIM, flight plans may be submitted 2 hours before the flight.}
    \end{enumempty}

    \itemsc{During flight}
    \begin{enumerate}
        \item A flight plan to be submitted during flight should normally be transmitted to the ATS unit in charge of the FIR, control area, advisory area or advisory route in or on which the aircraft is flying, or in or through which the aircraft wishes to fly or to the aeronautical telecommunication station serving the air traffic services unit concerned. When this is not practicable, it should be transmitted to another ATS unit or aeronautical telecommunication station for retransmission as required to the appropriate air traffic services unit.
        \item Where relevant, such as in respect of ATC units serving high- or medium-density airspace, the appropriate ATS authority should prescribe conditions and/or limitations with respect to the submission of flight plans during flight to ATC units.
        \note{If the flight plan is submitted for the purpose of obtaining air traffic control service, the aircraft is required to wait for an air traffic control clearance prior to proceeding under the conditions requiring compliance with air traffic control procedures. If the flight plan is submitted for the purpose of obtaining air traffic advisory service, the aircraft is required to wait for acknowledgment of receipt by the unit providing the service.}
    \end{enumerate}
\end{enumeratesc}

\subsubsection{Acceptance of a flight plan}

The first ATS unit receiving a flight plan, or change thereto, shall:

\begin{enumalph}
    \item check it for compliance with the format and data conventions;
    \item check it for completeness and, to the extent possible, for accuracy;
    \item take action, if necessary, to make it acceptable to the air traffic services; and
    \item indicate acceptance of the flight plan or change thereto, to the originator.
\end{enumalph}

\subsection[Air traffic control clearances]{AIR TRAFFIC CONTROL CLEARANCES}

\subsubsection{Scope and purpose}

\begin{enumerate}
    \item Clearances are issued solely for expediting and separating air traffic and are based on known traffic conditions which affect safety in aircraft operation. Such traffic conditions include not only aircraft in the air and on the manoeuvring area over which control is being exercised, but also any vehicular traffic or other obstructions not permanently installed on the manoeuvring area in use.
    \item If an air traffic control clearance is not suitable to the pilot-in-command of an aircraft, the flight crew may request and, if practicable, obtain an amended clearance.
    \item The issuance of air traffic control clearances by air traffic control units constitutes authority for an aircraft to proceed only in so far as known air traffic is concerned. ATC clearances do not constitute authority to violate any applicable regulations for promoting the safety of flight operations or for any other purpose; neither do clearances relieve a pilot-in-command of any responsibility whatsoever in connection with a possible violation of applicable rules and regulations.
    \item ATC units shall issue such ATC clearances as are necessary to prevent collisions and to expedite and maintain an orderly flow of air traffic.
    \item ATC clearances must be issued early enough to ensure that they are transmitted to the aircraft in sufficient time for it to comply with them.
\end{enumerate}

\subsubsection{Aircraft subject to ATC for part of flight}

\begin{enumerate}
    \item When a flight plan specifies that the initial portion of a flight will be uncontrolled, and that the subsequent portion of the flight will be subject to ATC, the aircraft shall be advised to obtain its clearance from the ATC unit in whose area controlled flight will be commenced.
    \item When a flight plan specifies that the first portion of a flight will be subject to ATC, and that the subsequent portion will be uncontrolled, the aircraft shall normally be cleared to the point at which the controlled flight terminates.
\end{enumerate}

%Numbering
% 4.5.3 Flights through intermediate stops

\subsubsection{Contents of clearances}

\begin{enumerate}
    \item Clearances shall contain positive and concise data and shall, as far as practicable, be phrased in a standard manner.
    
    %References
    \item Clearances shall, except as provided for in Chapter 6, Section 6.3.2, concerning standard departure clearances, contain the items specified in Chapter 11, 11.4.2.6.2.1.
\end{enumerate}

\subsubsection{Departing aircraft}

ACCs shall, except where procedures providing for the use of standard departure clearances have been implemented, forward a clearance to approach control units or aerodrome control towers with the least possible delay after receipt of request made by these units, or prior to such request if practicable.

\subsubsection{En-route aircraft}

\begin{enumeratesc}
    \itemsc{General}
    \begin{enumerate}
        \item An ATC unit may request an adjacent ATC unit to clear aircraft to a specified point during a specified period.
        \item After the initial clearance has been issued to an aircraft at the point of departure, it will be the responsibility of the appropriate ATC unit to issue an amended clearance whenever necessary and to issue traffic information, if required.
        \item When so requested by the flight crew, an aircraft shall be cleared for cruise climb whenever traffic conditions and coordination procedures permit. Such clearance shall be for cruise climb either above a specified level or between specified levels.
    \end{enumerate}

    \itemsc{Clearances relating to supersonic flight}
    \begin{enumerate}
        \item Aircraft intending supersonic flight shall, whenever practicable, be cleared for the transonic acceleration phase prior to departure.
        \item During the transonic and supersonic phases of a flight, amendments to the clearance should be kept to a minimum and must take due account of the operational limitations of the aircraft in these flight phases.
    \end{enumerate}
\end{enumeratesc}

\subsubsection{Description of air traffic control clearances}

\begin{enumeratesc}
    \itemsc{Clearance limit}
    \begin{enumerate}
        \item A clearance limit shall be described by specifying the name of the appropriate significant point, or aerodrome, or controlled airspace boundary.
        \item When prior coordination has been effected with units under whose control the aircraft will subsequently come, or if there is reasonable assurance that it can be effected a reasonable time prior to their assumption of control, the clearance limit shall be the destination aerodrome or, if not practicable, an appropriate intermediate point, and coordination shall be expedited so that a clearance to the destination aerodrome may be issued as soon as possible.
        \item If an aircraft has been cleared to an intermediate point in adjacent controlled airspace, the appropriate ATC unit will then be responsible for issuing, as soon as practicable, an amended clearance to the destination aerodrome.
        \item When the destination aerodrome is outside controlled airspace, the ATC unit responsible for the last controlled airspace through which an aircraft will pass shall issue the appropriate clearance for flight to the limit of that controlled airspace.
    \end{enumerate}

    \itemsc{Route of flight}
    \begin{enumerate}
        \item The route of flight shall be detailed in each clearance when deemed necessary. The phrase ``cleared flight planned route" may be used to describe any route or portion thereof, provided the route or portion thereof is identical to that filed in the flight plan and sufficient routing details are given to definitely establish the aircraft on its route. The phrases ``cleared (designation) departure" or ``cleared (designation) arrival" may be used when standard departure or arrival routes have been established by the appropriate ATS authority and published in Aeronautical Information Publications (AIPs).
        %References
        \note{See 6.3.2.3 pertaining to standard clearances for departing aircraft and 6.5.2.3 pertaining to standard clearances for arriving aircraft.}
        \item The phrase ``cleared flight planned route" shall not be used when granting a re-clearance.
        \item Subject to airspace constraints, ATC workload and traffic density, and provided coordination can be effected in a timely manner, an aircraft should whenever possible be offered the most direct routing.
    \end{enumerate}

    \itemsc{Levels}
    %References
    \begin{enumempty}
        \item Except as provided for in Chapter 6, 6.3.2 and 6.5.1.5, use of standard departure and arrival clearances, instructions included in clearances relating to levels shall consist of the items specified in Chapter 11, 11.4.2.6.2.2.
    \end{enumempty}

    \itemsc{Clearance of a requested change in flight plan}
    \begin{enumerate}
        \item When issuing a clearance covering a requested change in route or level, the exact nature of the change shall be included in the clearance.
        \item \label{4.5.7.4.2} When traffic conditions will not permit clearance of a requested change, the word ``UNABLE" shall be used. When warranted by circumstances, an alternative route or level should be offered.
        \item When an alternative route is offered and accepted by the flight crew under the procedures described in \ref{4.5.7.4.2}, the amended clearance issued shall describe the route to the point where it joins the previously cleared route, or, if the aircraft will not re-join the previous route, to the destination.
    \end{enumerate}

    \itemsc{Readback of clearances}
    \begin{enumerate}
        \item The flight crew shall read back to the air traffic controller safety-related parts of ATC clearances and instructions which are transmitted by voice. The following items shall always be read back:
        \begin{enumalph}
            \item ATC route clearances;
            \item clearances and instructions to enter, land on, take off from, hold short of, cross, taxi and backtrack on any runway; and
            \item runway-in-use, altimeter settings, SSR codes, level instructions, heading and speed instructions and, whether issued by the controller or contained in automatic terminal information service (ATIS) broadcasts, transition levels.
        \end{enumalph}

        \note{If the level of an aircraft is reported in relation to standard pressure 1 013.25 hPa, the words ``FLIGHT LEVEL" precede the level figures. If the level of the aircraft is reported in relation to QNH/QFE, the figures are followed by the word ``METRES" or ``FEET", as appropriate.}
        
        \begin{enumerate}
            \item Other clearances or instructions, including conditional clearances, shall be read back or acknowledged in a manner to clearly indicate that they have been understood and will be complied with.
        \end{enumerate}

        \item The controller shall listen to the readback to ascertain that the clearance or instruction has been correctly acknowledged by the flight crew and shall take immediate action to correct any discrepancies revealed by the readback.
        \begin{enumerate}
            \item Unless specified by the appropriate ATS authority, voice readback of controller-pilot data link communications (CPDLC) messages shall not be required.
        \end{enumerate}

        %References
        \note{The procedures and provisions relating to the exchange and acknowledgement of CPDLC messages are contained in Annex 10, Volume II and the PANS-ATM, Chapter 14.}
    \end{enumerate}
\end{enumeratesc}

\subsection[Horizontal speed control instructions]{HORIZONTAL SPEED CONTROL INSTRUCTIONS} \label{4.6}

\subsubsection{General}

\begin{enumerate}
    \item In order to facilitate a safe and orderly flow of traffic, aircraft may, subject to conditions specified by the appropriate authority, be instructed to adjust speed in a specified manner. Flight crews should be given adequate notice of planned speed control.
    \note[1]{Application of speed control over a long period of time may affect aircraft fuel reserves.}
    %References
    \note[2]{Provisions concerning longitudinal separation using the Mach number technique are contained in Chapter 5, Separation Methods and Minima.}
    \item Speed control instructions shall remain in effect unless explicitly cancelled or amended by the controller.
    \note{Cancellation of any speed control instruction does not relieve the flight crew of compliance with speed limitations associated with airspace classifications as specified in Annex 11 -- Air Traffic Services, Appendix 4.}
    \item Speed control shall not be applied to aircraft entering or established in a holding pattern.
    \item Speed adjustments should be limited to those necessary to establish and/or maintain a desired separation minimum or spacing. Instructions involving frequent changes of speed, including alternate speed increases and decreases, should be avoided.
    \item The flight crew shall inform the ATC unit concerned if at any time they are unable to comply with a speed instruction. In such cases, the controller shall apply an alternative method to achieve the desired spacing between the aircraft concerned.
    \item At levels at or above FL 250, speed adjustments should be expressed in multiples of 0.01 Mach; at levels below FL 250, speed adjustments should be expressed in multiples of 10 kt based on indicated airspeed (IAS).
    \note[1]{Mach 0.01 equals approximately 6 kt IAS at higher flight levels.}
    \note[2]{When an aircraft is heavily loaded and at a high level, its ability to change speed may, in cases, be very limited.}
    \item Aircraft shall be advised when a speed control restriction is no longer required.
\end{enumerate}

\subsubsection{Methods of application}

\begin{enumerate}
    \item In order to establish a desired spacing between two or more successive aircraft, the controller should first either reduce the speed of the last aircraft, or increase the speed of the lead aircraft, then adjust the speed(s) of the other aircraft in order.
    \item In order to maintain a desired spacing using speed control techniques, specific speeds need to be assigned to all the aircraft concerned.
    \note[1]{The true airspeed (TAS) of an aircraft will decrease during descent when maintaining a constant IAS. When two descending aircraft maintain the same IAS, and the leading aircraft is at the lower level, the TAS of the leading aircraft will be lower than that of the following aircraft. The distance between the two aircraft will thus be reduced, unless a sufficient speed differential is applied. For the purpose of calculating a desired speed differential between two succeeding aircraft, 6 kt IAS per 1 000 ft height difference may be used as a general rule. At levels below FL 80 the difference between IAS and TAS is negligible for speed control purposes.}
    \note[2]{Time and distance required to achieve a desired spacing will increase with higher levels, higher speeds, and when the aircraft is in a clean configuration.}
\end{enumerate}

%Numbering
%4.5.3 Flights through intermediate stops

\subsubsection{Descending and arriving aircraft}

\begin{enumerate}
    \item An aircraft should, when practicable, be authorized to absorb a period of notified terminal delay by cruising at a reduced speed for the latter portion of its flight.
    \item An arriving aircraft may be instructed to maintain its ``maximum speed", ``minimum clean speed", ``minimum speed", or a specified speed.
    \note{``Minimum clean speed" signifies the minimum speed at which an aircraft can be flown in a clean configuration, i.e. without deployment of lift-augmentation devices, speed brakes or landing gear.}
    \item Speed reductions to less than 250 kt IAS for turbojet aircraft during initial descent from cruising level should be applied only with the concurrence of the flight crew.
    \item Instructions for an aircraft to simultaneously maintain a high rate of descent and reduce its speed should be avoided as such manoeuvres are normally not compatible. Any significant speed reduction during descent may require the aircraft to temporarily level off to reduce speed before continuing descent.
    \item Arriving aircraft should be permitted to operate in a clean configuration for as long as possible. Below FL 150, speed reductions for turbojet aircraft to not less than 220 kt IAS, which will normally be very close to the minimum speed of turbojet aircraft in a clean configuration, may be used.
    \item Only minor speed adjustments not exceeding plus/minus 20 kt IAS should be used for aircraft on intermediate and final approach.
    \item Speed control should not be applied to aircraft after passing a point 4 NM from the threshold on final approach.
    %References
    \note{The flight crew has a requirement to fly a stabilized approach (airspeed and configuration) typically by 3 NM from the threshold (Doc 8168, PANS-OPS, Volume I, Part III, Section 4, Chapter 3, 3.3 refers).}
\end{enumerate}

\subsubsection{SID and STAR}

The flight crew shall comply with published SID and STAR speed restrictions unless the restrictions are explicitly cancelled or amended by the controller.
\note[1]{Some SID and STAR speed restrictions ensure containment within RNAV departure or arrival procedure (e.g. maximum speed associated with a constant radius arc to a fix (RF) leg).}
%References
\note[2]{See 6.3.2.4 pertaining to clearances on a SID and 6.5.2.4 pertaining to clearances on a STAR.}

\subsection[Vertical speed control instructions]{VERTICAL SPEED CONTROL INSTRUCTIONS}

\subsubsection{General}

\begin{enumerate}
    \item In order to facilitate a safe and orderly flow of traffic, aircraft may be instructed to adjust rate of climb or rate of descent. Vertical speed control may be applied between two climbing aircraft or two descending aircraft in order to establish or maintain a specific vertical separation minimum.
    \item Vertical speed adjustments should be limited to those necessary to establish and/or maintain a desired separation minimum. Instructions involving frequent changes of climb/descent rates should be avoided.
    \item The flight crew shall inform the ATC unit concerned if unable, at any time, to comply with a specified rate of climb or descent. In such cases, the controller shall apply an alternative method to achieve an appropriate separation minimum between aircraft, without delay.
    \item Aircraft shall be advised when a rate of climb/descent restriction is no longer required.
\end{enumerate}

\subsubsection{Methods of application}

\begin{enumerate}
    \item An aircraft may be instructed to expedite climb or descent as appropriate to or through a specified level, or may be instructed to reduce its rate of climb or rate of descent.
    \item Climbing aircraft may be instructed to maintain a specified rate of climb, a rate of climb equal to or greater than a specified value or a rate of climb equal to or less than a specified value.
    \item Descending aircraft may be instructed to maintain a specified rate of descent, a rate of descent equal to or greater than a specified value or a rate of descent equal to or less than a specified value.
    \item In applying vertical speed control, the controller should ascertain to which level(s) climbing aircraft can sustain a specified rate of climb or, in the case of descending aircraft, the specified rate of descent which can be sustained, and shall ensure that alternative methods of maintaining separation can be applied in a timely manner, if required.
    \note{Controllers need to be aware of aircraft performance characteristics and limitations in relation to a simultaneous application of horizontal and vertical speed limitations.}
\end{enumerate}

\subsection[Change from IFR to VFR flight]{CHANGE FROM IFR TO VFR FLIGHT}

\begin{enumnoss}
    \item Change from instrument flight rules (IFR) flight to visual flight rules (VFR) flight is only acceptable when a message initiated by the pilot-in-command containing the specific expression ``CANCELLING MY IFR FLIGHT", together with the changes, if any, to be made to the current flight plan, is received by an air traffic services unit. No invitation to change from IFR flight to VFR flight is to be made either directly or by inference.
    \item No reply, other than the acknowledgment ``IFR FLIGHT CANCELLED AT ... (time)", should normally be made by an air traffic services unit.
    \item When an ATS unit is in possession of information that instrument meteorological conditions are likely to be encountered along the route of flight, a pilot changing from IFR flight to VFR flight should, if practicable, be so advised.
    %References
    \note{See Chapter 11, 11.4.3.2.1.}
    \item An ATC unit receiving notification of an aircraft's intention to change from IFR to VFR flight shall, as soon as practicable thereafter, so inform all other ATS units to whom the IFR flight plan was addressed, except those units through whose regions or areas the flight has already passed.
\end{enumnoss}

\subsection[Wake turbulence categories]{WAKE TURBULENCE CATEGORIES} \label{4.9}

%References
\note{The term ``wake turbulence" is used in this context to describe the effect of the rotating air masses generated behind the wing tips of large jet aircraft, in preference to the term ``wake vortex" which describes the nature of the air masses. Detailed characteristics of wake vortices and their effect on aircraft are contained in the \textup{Air Traffic Services Planning Manual} (Doc 9426), Part II, Section 5.}

\subsubsection{Wake turbulence categories of aircraft}

\begin{enumerate}
    \item Wake turbulence separation minima shall be based on a grouping of aircraft types into three categories according to the maximum certificated take-off mass as follows:

    \begin{enumalph}
        \item HEAVY (H) -- all aircraft types of 136 000 kg or more;
        \item MEDIUM (M) -- aircraft types less than 136 000 kg but more than 7 000 kg; and
        \item LIGHT (L) -- aircraft types of 7 000 kg or less.
    \end{enumalph}

    \item Helicopters should be kept well clear of light aircraft when hovering or while air taxiing.
    \note[1]{Helicopters produce vortices when in flight and there is some evidence that, per kilogram of gross mass, their vortices are more intense than those of fixed-wing aircraft.}
    %References
    \note[2]{The provisions governing wake turbulence separation minima are set forth in Chapter 5, Section 5.8, and Chapter 8, Section 8.7.3.}
\end{enumerate}

\subsubsection{Indication of heavy wake turbulence category}

For aircraft in the heavy wake turbulence category the word ``Heavy" shall be included immediately after the aircraft call sign in the initial radiotelephony contact between such aircraft and ATS units.
%References
\note{Wake turbulence categories are specified in the instructions for completing Item 9 of the flight plan in Appendix 2.}

\subsection[Altimeter setting procedures]{ALTIMETER SETTING PROCEDURES}

\subsubsection{Expression of vertical position of aircraft}

\begin{enumerate}
    \item \label{4.10.1.1} For flights in the vicinity of aerodromes and within terminal control areas the vertical position of aircraft shall, except as provided for in \ref{4.10.1.2}, be expressed in terms of altitudes at or below the transition altitude and in terms of flight levels at or above the transition level. While passing through the transition layer, vertical position shall be expressed in terms of flight levels when climbing and in terms of altitudes when descending.
    \item \label{4.10.1.2} When an aircraft which has been given clearance to land is completing its approach using atmospheric pressure at aerodrome elevation (QFE), the vertical position of the aircraft shall be expressed in terms of height above aerodrome elevation during that portion of its flight for which QFE may be used, except that it shall be expressed in terms of height above runway threshold elevation:

    \begin{enumalph}
        \item for instrument runways, if the threshold is 7 ft (2 m) or more below the aerodrome elevation; and
        \item for precision approach runways.
    \end{enumalph}

    \item For flights en route, the vertical position of aircraft shall be expressed in terms of:

    \begin{enumalph}
        \item flight levels at or above the lowest usable flight level; and
        \item altitudes below the lowest usable flight level;
    \end{enumalph}

    \noindent except where, on the basis of regional air navigation agreements, a transition altitude has been established for a specified area, in which case the provisions of \ref{4.10.1.1} shall apply.
\end{enumerate}

\subsubsection{Determination of the transition level}

\begin{enumerate}
    \item The appropriate ATS unit shall establish the transition level to be used in the vicinity of the aerodrome(s) concerned and, when relevant, the terminal control area (TMA) concerned, for the appropriate period of time on the basis of QNH (altimeter subscale setting to obtain elevation when on the ground) reports and forecast mean sea level pressure, if required.
    \item The transition level shall be the lowest flight level available for use above the transition altitude established for the aerodrome(s) concerned. Where a common transition altitude has been established for two or more aerodromes which are so closely located as to require coordinated procedures, the appropriate ATS units shall establish a common transition level to be used at any given time in the vicinity of the aerodrome and, when relevant, in the TMA concerned.
    \note{See \ref{4.10.3.2} regarding the determination of the lowest usable flight level(s) for control areas.}
\end{enumerate}

\subsubsection{Minimum cruising level for IFR flights}

\begin{enumerate}
    \item Except when specifically authorized by the appropriate authority, cruising levels below the minimum flight altitudes established by the State shall not be assigned.
    \item \label{4.10.3.2} ATC units shall, when circumstances warrant it, determine the lowest usable flight level or levels for the whole or parts of the control area for which they are responsible, use it when assigning flight levels and pass it to pilots on request.
    \note[1]{Unless otherwise prescribed by the State concerned, the lowest usable flight level is that flight level which corresponds to, or is immediately above, the established minimum flight altitude.}
    \note[2]{The portion of a control area for which a particular lowest usable flight level applies is determined in accordance with air traffic services requirements.}
    %References
    \note[3]{The objectives of the air traffic control service as prescribed in Annex 11 do not include prevention of collision with terrain. The procedures prescribed in this document do not relieve pilots of their responsibility to ensure that any clearances issued by air traffic control units are safe in this respect. When an IFR flight is vectored or is given a direct routing which takes the aircraft off an ATS route, the procedures in Chapter 8, 8.6.5.2 apply.}
\end{enumerate}

\subsubsection{Provision of altimeter setting information}

\begin{enumerate}
    \item Appropriate ATS units shall at all times have available for transmission to aircraft in flight, on request, the information required to determine the lowest flight level which will ensure adequate terrain clearance on routes or segments of routes for which this information is required.
    \note{If so prescribed on the basis of regional air navigation agreements, this information may consist of climatological data.}
    \item Flight information centres and ACCs shall have available for transmission to aircraft, on request, an appropriate number of QNH reports or forecast pressures for the FIRs and control areas for which they are responsible, and for those adjacent.
    \item The flight crew shall be provided with the transition level in due time prior to reaching it during descent. This may be accomplished by voice communications, ATIS broadcast or data link.
    \item The transition level shall be included in approach clearances when so prescribed by the appropriate authority or requested by the pilot.
    \item A QNH altimeter setting shall be included in the descent clearance when first cleared to an altitude below the transition level, in approach clearances or clearances to enter the traffic circuit, and in taxi clearances for departing aircraft, except when it is known that the aircraft has already received the information.
    \item A QFE altimeter setting shall be provided to aircraft on request or on a regular basis in accordance with local arrangements; it shall be the QFE for the aerodrome elevation except for:

        \begin{enumalph}
        \item for instrument runways, if the threshold is 7 ft (2 m) or more below the aerodrome elevation; and
        \item for precision approach runways;
    \end{enumalph}

    \noindent in which cases the QFE for the relevant runway threshold shall be provided.

    \item Altimeter settings provided to aircraft shall be rounded down to the nearest lower whole hectopascal.
    \note[1]{Unless otherwise prescribed by the State concerned, the lowest usable flight level is that flight level which corresponds to, or is immediately above, the established minimum flight altitude.}
    \note[2]{The portion of a control area for which a particular lowest usable flight level applies is determined in accordance with air traffic services requirements.}
    %References
    \note[3]{See Foreword, Note 2 to paragraph 2.1.}
\end{enumerate}

\subsection[Position reporting]{POSITION REPORTING}

\subsubsection{Transmission of position reports}

\begin{enumerate}
    \item \label{4.11.1.1} On routes defined by designated significant points, position reports shall be made by the aircraft when over, or as soon as possible after passing, each designated compulsory reporting point, except as provided in \ref{4.11.1.3} and \ref{4.11.3}. Additional reports over other points may be requested by the appropriate ATS unit.
    \item \label{4.11.1.2} On routes not defined by designated significant points, position reports shall be made by the aircraft as soon as possible after the first half hour of flight and at hourly intervals thereafter, except as provided in \ref{4.11.1.3}. Additional reports at shorter intervals of time may be requested by the appropriate ATS unit.
    \item \label{4.11.1.3} Under conditions specified by the appropriate ATS authority, flights may be exempted from the requirement to make position reports at each designated compulsory reporting point or interval. In applying this, account should be taken of the meteorological requirement for the making and reporting of routine aircraft observations.
    %References
    \note{This is intended to apply in cases where adequate flight progress data are available from other sources, e.g. radar or ADS-B (see Chapter 8, 8.6.4.4), or ADS-C (see Chapter 13) and in other circumstances where the omission of routine reports from selected flights is found to be acceptable.}
    \item The position reports required by \ref{4.11.1.1} and \ref{4.11.1.2} shall be made to the ATS unit serving the airspace in which the aircraft is operated. In addition, when so prescribed by the appropriate ATS authority in aeronautical information publications or requested by the appropriate ATS unit, the last position report before passing from one FIR or control area to an adjacent FIR or control area shall be made to the ATS unit serving the airspace about to be entered.
    \item If a position report is not received at the expected time, subsequent control shall not be based on the assumption that the estimated time is accurate. Immediate action shall be taken to obtain the report if it is likely to have any bearing on the control of other aircraft.
\end{enumerate}

\subsubsection{Contents of voice position reports}

\begin{enumerate}
    \item The position reports required by \ref{4.11.1.1} and \ref{4.11.1.2} shall contain the following elements of information, except that elements \ref{4.11.2.1d}, \ref{4.11.2.1e} and \ref{4.11.2.1f} may be omitted from position reports transmitted by radiotelephony, when so prescribed on the basis of regional air navigation agreements:

    \begin{enumalph}
        \item aircraft identification;
        \item position;
        \item time;
        \item \label{4.11.2.1d} flight level or altitude, including passing level and cleared level if not maintaining the cleared level;
        \item \label{4.11.2.1e} next position and time over; and
        \item \label{4.11.2.1f} ensuing significant point.
    \end{enumalph}

    \begin{enumerate}
        \item Element \ref{4.11.2.1d}, flight level or altitude, shall, however, be included in the initial call after a change of air-ground voice communication channel.
    \end{enumerate}

    \item When assigned a speed to maintain, the flight crew shall include this speed in their position reports. The assigned speed shall also be included in the initial call after a change of air-ground voice communication channel, whether or not a full position report is required.
    \note{Omission of element \ref{4.11.2.1d} may be possible when flight level or altitude, as appropriate, derived from pressure-altitude information can be made continuously available to controllers in labels associated with the position indication of aircraft and when adequate procedures have been developed to guarantee the safe and efficient use of this altitude information.}
\end{enumerate}

\subsubsection[Radiotelephony procedures for air-ground voice communication channel changeover]{Radiotelephony procedures for air-ground \\ voice communication channel changeover} \label{4.11.3}

When so prescribed by the appropriate ATS authority, the initial call to an ATC unit after a change of air-ground voice communication channel shall contain the following elements:

\begin{enumalph}
    \item designation of the station being called;
    \item call sign and, for aircraft in the heavy wake turbulence category, the word ``Heavy";
    \item level, including passing and cleared levels if not maintaining the cleared level;
    \item speed, if assigned by ATC; and
    \item additional elements, as required by the appropriate ATS authority.
\end{enumalph}

%Numbering
%4.11.14
%4.11.15
%4.11.16

%4.12 REPORTING OF OPERATIONAL AND METEOROLOGICAL INFORMATION

\subsection[Presentation and updating of flight plan and control data]{PRESENTATION AND UPDATING OF \\ FLIGHT PLAN AND CONTROL DATA}

\subsubsection{General}

The appropriate authority shall establish provisions and procedures for the presentation to controllers, and subsequent updating, of flight plan and control data for all flights being provided with a service by an ATS unit. Provision shall also be made for the presentation of any other information required or desirable for the provision of ATS.

\subsubsection{Information and data to be presented}

\begin{enumerate}
    \item Sufficient information and data shall be presented in such a manner as to enable the controller to have a complete representation of the current air traffic situation within the controller's area of responsibility and, when relevant, movements on the manoeuvring area of aerodromes. The presentation shall be updated in accordance with the progress of aircraft, in order to facilitate the timely detection and resolution of conflicts as well as to facilitate and provide a record of coordination with adjacent ATS units and control sectors.
    \item An appropriate representation of the airspace configuration, including significant points and information related to such points, shall be provided. Data to be presented shall include relevant information from flight plans and position reports as well as clearance and coordination data. The information display may be generated and updated automatically, or the data may be entered and updated by authorized personnel.
    \item Requirements regarding other information to be displayed, or to be available for display, shall be specified by the appropriate authority.
\end{enumerate}

\subsubsection{Presentation of information and data}

\begin{enumerate}
    \item The required flight plan and control data may be presented through the use of paper flight progress strips or electronic flight progress strips, by other electronic presentation forms or by a combination of presentation methods.
    \item The method(s) of presenting information and data shall be in accordance with Human Factors principles. All data, including data related to individual aircraft, shall be presented in a manner minimizing the potential for misinterpretation or misunderstanding.
    \item Means and methods for manually entering data in ATC automation systems shall be in accordance with Human Factors principles.
    \item When flight progress strips (FPS) are used, there should be at least one individual FPS for each flight. The number of FPS for individual flights shall be sufficient to meet the requirements of the ATS unit concerned. Procedures for annotating data and provisions specifying the types of data to be entered on FPS, including the use of symbols, shall be specified by the appropriate ATS authority.
    \note{Guidance material on the use of paper FPS is contained in the Air Traffic Services Planning Manual (Doc 9426).}
    \item Data generated automatically shall be presented to the controller in a timely manner. The presentation of information and data for individual flights shall continue until such time as the data is no longer required for the purpose of providing control, including conflict detection and the coordination of flights, or until terminated by the controller.
\end{enumerate}

\subsection[Failure of irregularity of systems and equipment]{FAILURE OR IRREGULARITY OF \\ SYSTEMS AND EQUIPMENT}

ATC units shall immediately report in accordance with local instructions any failure or irregularity of communication, navigation and surveillance systems or any other safety-significant systems or equipment which could adversely affect the safety or efficiency of flight operations and/or the provision of air traffic control service.

\subsection[Data link communications initiation procedures]{DATA LINK COMMUNICATIONS INITIATION PROCEDURES}

\subsubsection{General}

%References
\note[1]{Provisions concerning the data link initiation capability (DLIC) are contained in Annex 10, Volume II, Chapter 8.}
\note[2]{Guidance material relating to the implementation of DLIC can be found in the \textup{Global Operational Data Link (GOLD) Manual} (Doc 10037).}

\begin{enumerate}        
    \item Before entering an airspace where data link applications are used by the ATS unit, data link communications shall be initiated between the aircraft and the ATS unit in order to register the aircraft and, when necessary, allow the start of a data link application. This shall be initiated by the aircraft, either automatically or by the pilot, or by the ATS unit on address forwarding.
    \item The logon address associated with an ATS unit shall be published in Aeronautical Information Publications in accordance with Annex 15.
    \note{A given FIR may have multiple logon addresses; and more than one FIR may share the same logon address.}
\end{enumerate}

\subsubsection{Aircraft initiation}

On receipt of a valid data link initiation request from an aircraft approaching or within a data link service area, the ATS unit shall accept the request and, if able to correlate it with a flight plan, shall establish a connection with the aircraft.

\subsubsection{ATS unit forwarding}

Where the ground system initially contacted by the aircraft is able to pass the necessary aircraft address information to another ATS unit, it shall pass the aircraft updated ground addressing information for data link applications previously coordinated in sufficient time to permit the establishment of data link communications.

%4.15.4

\chapterend